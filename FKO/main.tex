
         %% %% %% %%         Preamble & Options          %% %% %% %%
%~~~~~~~~~~~~~~~~~~~~~~~~~~~~~~~~~~~~~~~~~~~~~~~~~~~~~~~~~~~~~~~~~~~~~~~~~~~~~~%
\documentclass[paper=a4,fontsize=10pt,DIV=18,twocolumn,parskip=half]{scrartcl}
%%%%%%%%%%%%                      PACKAGES                           %%%%%%%%%%%
%~~~~~~~~~~~~~~~~~~~~~~~~~~~~~~~~~~~~~~~~~~~~~~~~~~~~~~~~~~~~~~~~~~~~~~~~~~~~~~%
\usepackage[utf8]{inputenc}		% utf8 encoding für umlaute und sonderzeichen
\usepackage[ngerman,english]{babel} 	% hyphenations in  different languages
\usepackage[tbtags]{amsmath}    % tbtags sets the number of split 
                                % environments at the end 
\usepackage{amssymb} 	    	% symbols
\usepackage{amsfonts} 	    	% font
\usepackage{textcomp} 		    % degree signs -> \textdegree
\usepackage{siunitx}		    % correct units and spacing
\usepackage{hyperref}	        % references and links
\usepackage{graphicx}           % embedd images
\usepackage{tikz}			    % tikz painting, e.g circled text
\usepackage{subfig}				% subfigure for mutliple pictures
\usepackage{color}              % define own colors
\usepackage[font=small,labelfont=bf,format=plain,margin=10pt]{caption}
\usepackage[noabbrev]{cleveref}
\usepackage{bijan_commands}     % self defined commands and shorthand notations
                                % !! has to be added manually !!


%\usepackage{pdfsync}		    % pdf synchronization (reverse search)
%\usepackage{booktabs}	        % nice table lines
%\usepackage{fancyhdr} 			% headers and footers
%\usepackage{fancybox} 			% boxes with rounded corners 
%\usepackage{wrapfig} 		    % figures floating beside the text
%\usepackage{picinpar}			% textbox beside the text
%\usepackage{placeins}			% float barriers in sections
%\usepackage{moreverb}
%\usepackage{listings}
%\usepackage{pifont} 			% DING fonts (used in the acknowledgement)
%\usepackage{scrtime}           % current time 
%\usepackage[numbers,sort&compress]{natbib} 		
                                % sorted references  with clickable URL 
%\usepackage{makeidx} 			% index at the end
%\usepackage[chapter,numbib]{tocbibind}     % add several things to the TOC


%%%%%%%%%%%                        SETUP                             %%%%%%%%%%%
%~~~~~~~~~~~~~~~~~~~~~~~~~~~~~~~~~~~~~~~~~~~~~~~~~~~~~~~~~~~~~~~~~~~~~~~~~~~~~~%
% color:
\definecolor{darkblue}{rgb}{0,0,0.5}
\definecolor{lila}{rgb}{0.3,0,0.3}
\definecolor{turq}{rgb}{0,0.1,0.4}

% siunitx:
\sisetup	{separate-uncertainty, per-mode=fraction}

% hyperref:
\hypersetup{	pdftex,
                colorlinks=true,
                backref=page,
                linkcolor=darkblue,     % usual links
                filecolor=red,
                citecolor=turq,  	    % for bibliographic
                urlcolor=lila,  		% for Emails and URLs
                pdfpagelabels=true, 	% that the pagenumbering is ok 
                breaklinks=true,
                plainpages=false,
                bookmarks=true, 
                bookmarksnumbered=false
                %pdftitle={--title--},
                %pdfauthor={--author--},
                %pdfsubject={},
                %pdfkeywords={--kind-of-work--},
            }

% hyperref:
% set black for printing	
% \hypersetup{
    % colorlinks,
    % citecolor=black,
    % filecolor=black,
    % linkcolor=black,
    % urlcolor=black
    % }

%tocibind:
%\setcounter{tocdepth}{2} % change to 1 that not so deep (smaller TOC)
           % defines packages and setups
%~~~~~~~~~~~~~~~~~~~~~~~~~~~~~~~~~~~~~~~~~~~~~~~~~~~~~~~~~~~~~~~~~~~~~~~~~~~~~~%
\numberwithin{equation}{section}    % number equations after sections
\columnsep20pt                      % width between \twocolumns
\linespread{1.2}
\allowdisplaybreaks[1]              % permissiveness of page breaks in equations
                                    % 1 ="allow page breaks but avoid them" 
                                    % and 4="break whenever you want".

%%% Spacings:
\setlength{\headheight}{2.0\baselineskip}       
% Fixes the 'small headhight'

\renewcommand*{\chapterheadstartvskip}{\vspace{0\baselineskip}} 
% Spacing Pagehead-Headline. Standard: 2

\renewcommand*{\chapterheadendvskip}{\vspace{\baselineskip}}
% Spacing Headline-Text

%%%%%%%%% %%%%%%%%% %%%%%%%%% %%%%%%%%% %%%%%%%%% %%%%%%%%% %%%%%%%%% %%%%%%%%%%


\begin{document}

\title{Festkörperoptik}                  
\author{Guilherme Stein \& Ulrich Müller}         
\date{}                             % Turn off automatic date
\twocolumn[\begin{@twocolumnfalse}
\vspace{-3em}
\maketitle      
% =============================================================================
\begin{abstract}      
% =============================================================================
\vspace{-2em}
    \noindent {\small Mit einem Fourietransformationsinfrarotspektrometer (FTIR) wurden zwei Galliumarsenid und Silizumproben untersucht. Deren Bandlücke wurde im undotierten Fall zu $E_{\mathrm{g,GaAs}} = \SI{1.384}{\eV}$ und $E_{\mathrm{g,SI}}=\SI{1.108}{\eV}$ bestimmt. Für die dotierten Proben wurden die Dotierkonzentrationen zu $N_{GaAs}=\SI{0.9e18}{\per\centi\meter^3}$ und $N_{Si}=\SI{1.1e18}{\per\centi\meter^3}$ berechnet. Im Falle des indirekten Interbandübergangs konnte bei Silizium die Energie der am Übergang beteiligten Phononen zu $\SI{48}{\milli\eV}$ bestimmt werden. Deutlich konnte der Einfluss der Gasabsorption von Wasser und Kohlenstoffdioxid in der Raumluft nachgewiesen werden.   
    }

\vspace{1em}

  \noindent Betreuer: Dr. Christoph Brüne
  \hfill Experimentdurchführung: $10$. October 2013
  \begin{flushright}
    % Protokollabgabe: 12. Oktober 2012        
  \end{flushright}
\end{abstract}

\vspace{2em}
\end{@twocolumnfalse}
]


% =============================================================================

\section{Einleitung}

Halbleiter spielen heutzutage eine wichtige Rolle in vielen technischen 
Anwendungen. Die Bandlücke, ein Energiebereich in dem es keine 
quantenmechanischen Zustände gibt, macht diese Materialklasse interessant für 
viele Anwendungen in der Elektronik und der Optoelektronik. Silizium zählt dabei 
zu den indirekten Halbleitern und ist ein wichtiger Bestandteil vieler 
Metall-Oxid-Halbleiter Bauelemente. Für optisch aktive Halbleiterbauelemente 
werden in der Regel Halbleiter mit einem direkten Bandübergang verwendet, um 
eine hohe Wechselwirkung der Elektronen mit Licht zu ermöglichen. Genau diese 
Wechselwirkung ist für viele Bereiche der Physik von großem Interesse. \\
Ein etabliertes Werkzeug zur Charakterisierung von Halbleitern ist das 
Fourier-Transform-Infrarotspektrometers (FTIR). In diesem können Reflexions- und 
Transmissionsspektren einer Probe mit hoher Auflösung analysiert werden. Daraus 
lassen sich anschließend die Materialeigenschaften der Probe, wie komplexer 
Brechungsindex, Dotierkonzentrationen und Bandlücken berechnen. 

% =============================================================================

\section{Theorie}
In diesem Abschnitt werden die wichtigsten Zusammenhänge die für die spätere 
Betrachtung nötig sind vorstellen. Dabei orientieren wir uns an 
\cref{fkotheorie}.

\subsection{Dispersionsrelation und Brechungsindex}

Aus den Maxwellgleichungen lässt sich die Wellengleichung herleiten, welche für 
einen Ansatz ebener Wellen uns folgende Dispersionsrelation liefert:

\begin{align}
    q^2 = \frac{\omega^2}{c^2}\epsilon(\omega)
\end{align}
mit dem Wellenvektor $q$, der Frequenz $\omega$ und $c$, der 
Lichtgeschwindigkeit im Vakuum.
$\epsilon(\omega)=\epsilon'(\omega)+i\epsilon''(\omega)$ beschreibt dabei die 
komplexe Dielektrizitätsfunktion. 
Diese hängt mit dem komplexen Brechungsindex zusammen:

\begin{align}
    \sqrt{\epsilon}=n+i\kappa \\
    \epsilon'=n^2-\kappa^2 \\
    \epsilon''=2n\kappa
    \label{diel}
\end{align}

\subsection{Reflexion und Transmission}
Trifft Licht auf die Grenzfläche zwischen zwei unterschiedlichen Materialien, so 
wird ein gewisser Anteil der Intensität reflektiert. Ein anderer Teil des 
Lichtes gelangt in die Probe und kann dort absorbiert werden. Licht was nicht 
reflektiert oder absorbiert wird, verlässt die Probe als transmittiertes Licht 
wieder. Für die drei Koeffizienten diese drei Prozesse gilt deshalb

\begin{align}
    R+A+T=1.
\end{align}

Dabei sind die Koeffizienten eindeutig durch den komplexen Brechungsindex der 
Materialien festgelegt. Luft besitzt einen Brechungsindex von annähernd $1$.
Die Fresnelschen Formeln, die den Übergang des Lichtes von einem Medium in ein 
anderes beschreiben, vereinfachen sich bei senkrechtem Einfall deutlich. Man 
erhält für den Reflexionskoeffizienten

\begin{align}
    R=\left(\frac{1-N_2}{1+N_2}\right)^2.
\end{align}

und für den Transmissionskoeffizienten 

\begin{align}
    T=\left(\frac{2}{1+N_2}\right)^2.
\end{align}

\begin{figure}
    \centering
    \includegraphics[width=0.6\columnwidth]{Bilder/Reflexion.jpg}
    \caption{Mehrfachreflexion und -transmission an einer planparallelen 
    Schicht}
    \label{multi}
\end{figure}

Für eine planparallele Schicht muss man die wiederholte Reflexion und 
Transmission berücksichtigen, wie in \cref{multi} illustriert und erhält dabei 
dabei folgenden Ausdruck:
\begin{align}
    T_v= & \frac{(1-\rho)^2 \exp(-d\beta \nu )}{1-\rho^2 \exp (-2d\beta\nu)} \\ 
    \vspace{2em}
    R_v= & \rho+\frac{\rho (1-\rho)^2 \exp (-2 d \beta (\nu ))}{1-\rho^2 \exp(-2 
    d \beta \nu )}\\
    \vspace{2em}
    \text{wobei } \beta= & 4 \pi  \nu  \left| \kappa \right|
    \text{ und }\rho= \frac{\kappa ^2+(n-1)^2}{\kappa ^2+(n+1)^2}
    \label{trans}
\end{align}


\subsection{Lorentz Oszillator \& Drude Modell}

Aus dem Lorentz-Modell eines Elektrons als Oszillator erhält man eine 
dielektrische Funktion, die sich, obwohl durch klassische Überlegungen 
begründet, dem quantenmeschanischem Modell sehr nahe kommt. Dieses ist gegeben 
durch die Übergangsfrequenzen $\omega_j$, den Oszillatorstärken $f_j$ sowie den 
Inversen der Lebensdauern der Zustände $\Gamma_j$:

\begin{align}
    \epsilon(\omega) = 1 + \frac{e^2}{m\epsilon_0}\sum_j\frac{Nf_j}{\omega_j^2 
    - \omega^2 - i\Gamma_j\omega}
    \label{lorentz}
\end{align}

wobei die Summe aller Oszillatorstärken sich zu eins summiert.

In der Impulsdarstellung lassen sich diese als Matrixelemente des 
Impulsoperators darstellen:
\begin{align}
    f_{ij}=\frac{2}{m\hbar\omega_{ij}}\abs{p_{ij}}^2
\end{align}


Das Drude Modell betrachtet nun freie Ladungsträger mit verschwindenden 
Resonanzfrequenzen. Mit der sog. Plasmafrequenz 
$\omega_p=\frac{Ne^2}{m\epsilon_0}$ lässt sich \cref{lorentz} umformulieren zu:

\begin{align}
    \epsilon(\omega)=1-\frac{\omega_p\tau^2}{1+\omega^2\tau^2}+
    i\frac{\omega_p\tau}{\omega+\omega^3\tau^2}
    \label{drude}
\end{align}

% =============================================================================

\section{Experimenteller Aufbau}

Alle Reflexions und Transmissionsmessungen finden im einem  FTIR statt, das 
schematisch in \cref{ftir} dargestellt ist. Dabei wird das Licht, von einer 
Infrarot-Lichtquelle (Globar-Stab oder Wolfram-Lampe) durch ein Interferometer 
geleitet und interferiert in Abhängigkeit der Position des beweglichen Spiegels. 
Das Frequenzspektrum wird dadurch auf eine neue Funktion $I(L)$ abgebildet. 
Diese Funktion beschreibt die Abhängigkeit der detektierten Intensität von der 
Spiegelposition und ist mathematisch eine Fouriertransformation des 
Frequenzspektrums. Im optischen Strahlengang kann nun der Teil der Intensität 
gemessen werden, der reflektiert bzw. transmittiert wird. Nach der Messung führt 
das FTIR die inverse Fouriertransformation aus und man erhält wieder die  
Intensität in Abhängigkeit der Frequenz. Um den Einfluss der Probenreflexion und 
-transmission zu isolieren, wird zu jeder Messung eine Referenzmessung 
durchgeführt. Dies wird im Fall der Transmissionsmessung mit leerem Probenraum 
und im Fall der Reflexionsmessung mit einem Goldspiegel durchgeführt, der im 
untersuchten Frequenzbereich nahezu perfekt reflektiert.
Untersucht werden verschiedene Proben, die Gasabsorption an Luft und das 
Rauschen des Spektrometers.

Folgende Proben standen uns zur Verfügung:

\begin{tabular}{l | l c }
        Probe & Material & Schichtdicke   \\
        \hline
       \texttt{Sample \circled{1}} & GaAs undotiert & \SI{470}{\micro\meter} \\
       \texttt{Sample \circled{2}} & GaAs dotiert   & \SI{440}{\micro\meter} \\
       \texttt{Sample \circled{3}} & Si undotiert   & \SI{530}{\micro\meter} \\
       \texttt{Sample \circled{4}} & Si dotiert     & \SI{530}{\micro\meter} \\
    %\hline
    %lk  \\
    %& \SI{470}{\micro\meter} 
    %& \SI{440}{\micro\meter} 
    %& \SI{530}{\micro\meter} 
    %& \SI{530}{\micro\meter}  


\end{tabular}

\begin{figure}
\centering
    \includegraphics[width=0.4\textwidth]{Bilder/FTIR}
    \caption{Aufbau des FTIR. Hauptbestandteile sind Lichtquelle, 
    Interferometer, Probenhalter und Detektor.}
    \label{ftir}
\end{figure}

% =============================================================================

\section{Auswertung}

\subsection{Gasabsorption in Luft}
Am Anfang soll die Gasabsorption der Umgebungsluft untersucht werden.  Dazu 
verwenden wir die Globarquelle, die ein kontinuierliches Spektrum von 
$\SI{500}{\per\centi\meter}$ bis zu $\SI{5000}{\per\centi\meter}$ aussendet.  
Dieses Spektrum wird auf dem Weg zum Detektor von der Umgebungsluft an manchen 
Stellen absorbiert, was uns die Spektralbereiche starker Gasabsorption 
identifizieren lässt.
In Abbildung \ref{Luft1} ist das gesamte aufgenommene Spektrum gezeigt.  Wir können 
jeweils zwei Bereiche der Absorption von Wasser (\cref{Luft2},\cref{Luft3}) und 
Kohlenstoffdioxid (\cref{Luft4},\cref{Luft5}) zuordnen.  Deutlich zu erkennen sich 
dabei auch die im Fall von Kohlenstoffdioxid die P- und R-Zweige der 
Rotations-Schwingungs-Spektren.

\begin{figure}[htp]
	\begin{center}
		\includegraphics[width=\columnwidth]{Bilder/Luft1}
		\caption{Normierte Intensität der Globar-Quelle. Deutlich zu erkennen sind das Absoprtionsspektrum vom Kohlenstoffdioxid in rot und Wasser in blau. }
		\label{Luft1}
	\end{center}
\end{figure}

\begin{figure}
    \centering
    \begin{subfigure}{\columnwidth}
        \includegraphics[width=\textwidth]{Bilder/Luft2}
        \caption{CO2-Absorptionslinien}
        \label{Luft2}
    \end{subfigure}
    \begin{subfigure}{\columnwidth}
        \includegraphics[width=\textwidth]{Bilder/Luft3}
        \caption{CO2-Absorptionslinien}
        \label{Luft3}
    \end{subfigure}
    \begin{subfigure}{\columnwidth}
        \includegraphics[width=\textwidth]{Bilder/Luft4}
        \caption{Wasser-Absorptionslinien}
        \label{Luft4}
    \end{subfigure}
        \begin{subfigure}{\columnwidth}
        \includegraphics[width=\textwidth]{Bilder/Luft5}
        \caption{Wasser-Absorptionslinien}
        \label{Luft5}
    \end{subfigure}
    \caption{Die stärksten Absorptionslinien im Spektrum sind Wasser- und Kohlenstoffdioxid-Absorptionslinien. Bei den Absorptionslinien von CO2 ist die typische Form der Rotations-Schwingungs-Spektren zu erkennen.}
    \label{brechzahlen}
\end{figure}

\subsection{Signal Rausch Verhältnis}
Mit geringer Auflösung messen wir ohne eingebaute Probe das Spektrum der 
Globar-Quelle und normieren es auf eine zweite Messung mit gleichen 
Einstellungen. Diese Messung wiederholen wir mit $10$, $50$ und $100$ 
Spiegeldurchläufe (Scans) und untersuchen, wie das Rauschen mit zunehmender 
Anzahl an Scans abnimmt.\\
Die Rauschniveaus sind in Abbildung \ref{snr} geplottet und nehmen merklich mit längerer Messzeit ab.Zur Auswertung wählen wir einen Bereich in dem die Intensität der einzelnen 
Spektren ausreichend hoch ist. Das heißt einen Bereich ohne Absorptionslinien in 
der Nähe der maximalen Intensität ($\SI{1000}{\per\centi\meter}$ bis 
$\SI{2200}{\per\centi\meter}$). In diesem Bereich erhalten wir vom FTIR 
normierte Intensitätswerte, die um die $1$ schwanken. Aus diesen 
Intensitätswerten, bestimmen wir die Standardabweichung vom Mittelwert und 
interpretieren diese als Rauschen der Daten. Um zu untersuchen, ob das Rauschen 
einem $1/\sqrt{N}$-Zusammenhang folgt, plotten wir in \cref{snr3} das Rauschen 
über $1/\sqrt{N}$. Den Fehler der Standardabweichungen erhalten wir mit 
Gaußscher Fehlerfortpflanzung aus den Fehlern der einzelnen normierten 
Intensitätswerten, für die wir wiederum die Größe des Rauschen annehmen.
Da systematisch Fehler bei unserer Messmethode wegnormiert werden, sollten die Punkte Standardabweichung über $1/\sqrt{N}$ auf einer Ursprungsgeraden liegen. Zwar ist die Tendenz der abnehmenden Standardabweichung deutlich zu erkennen, die Messpunkte lassen sich aber knapp nicht im Rahmen ihrer Fehler mit einer Ursprungsgerade fitten.

\begin{figure}[htp]
	\begin{center}
		\includegraphics[width=\columnwidth]{Bilder/Rausch}
		\caption{Rauschniveaus von drei Messungen mit unterschiedlicher Messzeit. $N=10$ in rot, $N=50$ in grün und $N=100$ in blau. Die Datenpunkte wurden zur besseren Sichtbarkeit auf der $y$-Achse verschoben. }
		\label{snr}
	\end{center}
\end{figure}

\begin{figure}[htp]
	\begin{center}
		\includegraphics[width=\columnwidth]{Bilder/Rausch3}
		\caption{Rauschniveaus von drei Messungen mit unterschiedlicher Messzeit. $N=10$ in rot, $N=50$ in grün und $N=100$ in blau. Die Datenpunkte wurden zur besseren Sichtbarkeit auf der $y$-Achse verschoben. }
		\label{snr3}
	\end{center}
\end{figure}

\subsection{Brechzahl von GaAs und Si}
Die untersuchten Proben bestehen aus mehreren hundert Nanometer dicken 
Schichten.
Untersucht man das Reflexionsspektrum dieser Schichten, so kommt es zwischen den 
Grenzflächen dieser Schichten zur Interferenz, die sich als Modulation im 
Reflexionsspektrum bemerkbar machen.  Wir bestimmen an verschiedenen Stellen im 
Spektrum jeweils den Abstand von $50$ Interferenzmaxima. Die Ablesegenauigkeit 
der Peaks nehmen wir mit $\SI{\pm 1}{\centi\meter^{-1}}$ an.
Mit Hilfe der Probendicken lässt sich nun der der Realteil des Brechungsindexes 
der verschiedenen Materialien an verschiedenen Stellen im Spektrum bestimmen.

\begin{tabular}{ l | c c c }
  Wellen- & n (GaAs) & n(GaAs)&n(Si)\\
  zahl & undotiert & dotiert & undotiert \\
  \hline
  500 & 3.452(32) & - & 3.343(34) \\
  1000 & 3.245(28) & - & 3.348(34) \\
  1500 & 3.243(28) & - & 3.369(34) \\
  2000 & 3.224(28) & 3.303(27) & 3.363(34) \\
  2500 & 3.243(28) & 3.303(27) & 3.370(34) \\
  3000 & 3.263(28) & 3.284(27) & 3.394(35) \\
\end{tabular}

Die entsprechenden Dielektrizitätskonstanten ergeben sich aus dem Quadrat der 
komplexen Brechungsindizes, die jedoch mit dieser Methode nicht bestimmt werden können.

\subsection{Reflexion und Transmission der Proben}

\subsection{Brechzahlen, Extinktions- und Absorptionskoeffizienten}

Das OPUS Programm liefert uns eine Tabelle indem für die einzelnen Messungen das 
Spektrum als 2-Tupel von Intensität und Wellenzahl repräsentiert wird. Um einen 
größeren Frequenzbereich abzudecken, wurde bei der Messung jeweils ein Spektrum 
mit einem Globar-Stab und einer Wolfram Lampe durchgeführt. Aus den beiden 
Spektren der jeweiligen Probe, wird das Spektrum, welches zur weiteren 
Untersuchung benutzt wird, zusammengesetzt. Dabei wurden versucht, starke 
Oszillationen zu vermeiden. Diese treten auf, wenn die Intensität der Lampe in 
einem Frequenzbereich gering ist, und bei der Normierung das Rauschen einen 
erheblichen Beitrag zum Signal leistet. Auch wurde versucht die Stetigkeit des 
Spektrums zu erhalten. Um das Rauschen zu mininieren wurde eine Glättung 
durchgeführt und die resultierenden Punkte zu einer Funktion interpoliert. Diese 
wurden dann mit den theoretischen Werten aus \cref{trans} verglichen. Es wurde 
für die jeweilige Wellenzahl ein $n$ und $\kappa$ gefittet, welche die Differenz 
aus Theorie und experimentellen Daten verschwinden ließ. Somit erhält man 
$\kappa(\nu)$, $n(\nu)$ und $\beta(\nu)$ welche in \cref{brechzahlen} 
aufgetragen sind. 

\begin{figure}
    \centering
    \begin{subfigure}{\columnwidth}
        \includegraphics[width=\textwidth]{Bilder/brechzahl.pdf}
        \caption{Realteil vom Brechungsindex $n$}
        \label{bz}
    \end{subfigure}
    \begin{subfigure}{\columnwidth}
        \includegraphics[width=\textwidth]{Bilder/extinktion_avg.pdf}
        \caption{Imaginärteil vom Brechungsindex $\kappa$}
        \label{ex}
    \end{subfigure}
    \begin{subfigure}{\columnwidth}
        \includegraphics[width=\textwidth]{Bilder/absorption_avg.pdf}
        \caption{Absorptionskoeffizient $\beta$}
        \label{ab}
    \end{subfigure}
    \caption{Die verschiedenen Koeffizienten unserer untersuchten Proben.  
    \texttt{Sample 1} $\rightarrow$ \text{rot},
    \texttt{Sample 2} $\rightarrow$ \text{grün}, 
    \texttt{Sample 3} $\rightarrow$ \text{blau}}
    \label{brechzahlen}
\end{figure}

\subsection{Bandlücke und Impulsmatrixelemente}

Erhöht man die Energie, ausgehend von einer Wellenzahl unterhalb der Bandlücke, 
über die Energie der Bandlücke, so findet man, dass der Absorptionskoeffizient 
stark zunimmt. Der Grund dafür liegt in der Anzahl der Zustände, die oberhalb 
der Bandlücke einen Übergang ermöglichen. Im idealisierten Modell nimmt dabei 
die Zustandsdichte im dreidimensionalen System wurzelförmig zu und immer mehr 
Lorentzoszillatoren tragen zur Erhöhung des Absorptionskoeffizient bei.\\
Um die Bandlücke und das Impulsmatrixelement der Übergänge zu bestimmen, 
berechnen wir zunächst aus \cref{diel} den Imaginärteil $\epsilon^{``}$ des 
undotierten und dotierten GaAs. Nun tragen wir $(\epsilon^{``}\cdot \omega^2)^2$ 
über die Anregungswellenzahl in Abbildung \ref{??} auf. Der theoretische Verlauf 
dieser Funktion oberhalb der Bandlücke entspricht dabei

\begin{align}
   (\epsilon^{``}\cdot \omega^2)^2=\frac{e^4 (2\mu)^3 |p_{cv}|^4}{\epsilon_0^2 
   m_e^4 \hbar^2} \cdot (\hbar \omega - E_g)
\end{align}

Wir plotten zu dem Messpunkten eine theoretische Kurve, die unserer Meinung nach 
die Steigung nach dem Knick am besten wiedergibt und die Messpunkte etwa am 
steilsten Punkt tangentiert. Zu dieser Kurve plotten wir zwei Hilfsfunktionen, 
die die Messdaten etwas weiter ober und unten tangentieren, um eine Abschätzung 
der Unsicherheit von Bandlücke und Impulsmatrixelement zu erhalten. In allen 
folgenden Auswertung verwenden wir die Methode, die Fit-Paramtern wie hier leichtim Rahmen der experimentellen Daten zu variieren und damit den Fehler abzuschätzen.
Die Ergebnisse sind in Tabelle \ref{ime} dargestellt.

\begin{table}
    \begin{center}
\begin{tabular}[r]{ l | c c }
            &Bandlücke/ $\SI{}{\eV}	$		&Impulsmatrixelement/\\
  GaAs &  & $10^{-25}$ kg m/s\\
  \hline
  undotiert & 1.384(1) & 1.73(1) \\
 dotiert & 1.3591(5) & 1.59(3)   \end{tabular}
\caption{Ermittelte Werte für die Bandlücke von Galliumarsenid und entsprechende 
Impulsmatrixelement.}
  \label{ime}  \end{center}
\end{table}

Die von uns bestimmte Bandlücke von GaAs liegt mit $\SI{1.384}{\eV}$ um 
$\SI{50}{\milli\eV}$ unterhalb des Literaturwertes von 1.43 eV (bei 
$\SI{300}{\K}$). Der Grund liegt in der Aufweichung der Bandkante, dem 
sogenannten Urbachtail, der die Einflüsse von
Exzitonen, Phononen, Verunreinigungen und Defekte sowie die Dotierungen im 
Material berücksichtigt. Im Vergleich können wir erkennen, dass die Dotierung in 
unseren Proben zu einer Absenkung der Bandlücke führt.

\subsection{Bandlücke und charakteristische Phononenfrequenz für Silizium}

Bei Silizium handelt es sich um einen indirekten Halbleiter. Daher werden die 
Übergänge vom Valenz- zum Leitungsband maßgeblich von den Phononen im Material 
beeinflusst. Wir nutzen den Zusammenhang
\begin{align}
    \beta _{\text{em} }=\left(n_{\text{ph}}+1\right)\frac{\left(\hbar \omega 
    -\hbar \omega _{\text{ph}}-E_g\right){}^2}{(\hbar \omega )^2}
\end{align}

und 

\begin{align}
    \beta _{\text{abs} }=<n_{\text{ph}}>\frac{\left(\hbar \omega +\hbar \omega 
    _{\text{ph}}-E_g\right){}^2}{(\hbar \omega )^2}
\end{align}

der den Beitrag der Phononenabsorption und \mbox{-emission} beschreibt. Die 
theoretische Absorption wird dann zusammen mit den experimentellen Daten, als 
$\sqrt{\beta _{\text{em} }+\beta _{\text{abs} }}\nu $ linearisiert, über der 
Wellenzahl in \cref{??} geplottet. Der theoretische Verlauf wurde dabei durch 
Variation der Bandlücke und der Phononenenergie möglichst gut den 
experimentellen Daten angepasst. Die Werte für den besten Fit sind in \cref{phe} 
aufgelistet.
\begin{table}
    \begin{center}
\begin{tabular}{ l | c c }
  Material & Bandlücke/$\SI{}{\eV}$ & Phononenenergie/$\SI{}{\milli\eV}$\\
  \hline
  Si undotiert & 1.108(3) & 48(2)
\end{tabular}
  \caption{Ermittelte Werte für die Bandlücke von undotiertem Silizium und der 
  Phononenenergie der bei Übergang beteiligten Phononen.}
  \label{phe}
    \end{center}
\end{table}

Die Ermittelte Bandlücke von Silizium stimmt gut mit dem 
Literaturwert von $\SI{1.11}{\eV}$ \cite{kittel} (bei $\SI{300}{\K}$) überein.

\subsection{Beitrag der freien Ladungsträger zur Reflexion}

Um den Beitrag der freien Ladungsträger zur Reflektivität nachvollziehen zu 
können, plotten wir in \cref{DrudeRef} die Reflektivität im Drudemodell in 
Abhängigkeit der Wellenzahl für verschiedene Streuzeiten. Wir sehen für hohe 
Streuzeiten eine Reflektivität von nahezu $1$. An der Plasmakante bei $ 
w_p^2=\frac{N e^2}{\epsilon_0 m^*}$ fällt dann die Reflektivität asymptotisch 
auf null ab. Dies kann verstanden werden  wenn man die dielektrische Funktion 
betrachtet, die an der Plasmakante eine Nullstelle aufweist. Oberhalb und 
unterhalb der Plasmakante wechselwirkt die äußere elektromagnetische Welle sehr 
unterschiedlich mit dem Material und kann sich dadurch entweder im Material 
ausbreiten ($\epsilon > 0$) oder wird absorbiert ($\epsilon < 0$). Überhalb 
dieser Plasmakante schwingt das elektrische Feld schneller. Die Elektronen 
werden nur kurz durch das äußere Feld in eine Richtung beschleunigt und nehmen 
dadurch nur wenig Energie auf. Dadurch wechselwirkt die elektromagnetische Welle
nur wenig mit dem Material und kann dieses ungehindert durchdringen. Unterhalb 
werden die Elekronen lange in eine Richtung beschleunigt und nehmen viel 
Energie. Dies ist besonders dann effektiv, wenn die Elektronen wenig streuen, 
also lange Streuzeiten besitzen. In diesem Fall dringt die elektromagnetische 
Welle nur wenig in das Material ein und ein Großteil der Welle wird Reflektiert.

Im Halbleiter spielen sich grundsätzlich ähnliche Prozesse ab. Die wesentlichen 
Unterschiede hierbei sind jedoch, dass die Hintergrundsdielektrische-Funktion 
größer als $1$ ist, die Elektronen effektive Massen besitzen, die in 
unterschiedliche Raumrichtungen unterschiedlich groß sein können und dass das 
Gitter einen wesentlichen Beitrag zur dielektrischen Funktion spielen kann. In 
\ref{RefHalbleiter} ist zum Vergleich mit den Drudemodell eine Simulation mit 
$\epsilon_\infty=3.22^2$, einer effektiven Masse von $0.067m_e$ und einem 
Einfluss des Atomgitters geplottet.

\subsection{Drude Modell und charakteristische Parameter}

In den dotierten Proben wurde die Stromrelaxationszeit $\tau$, die 
Dotierkonzentration $\eta$ sowie die effektive Masse $m$ bestimmt. Die 
gemessenen Reflektionsspektren wurden an ein theoretisches Model mit 
entsprechenden freien Parametern angefittet. Der Fit berücksichtigt die 
Hochfrequenzleitfähigkeit nach Drude:

\begin{align}
    \sigma(\omega) = \frac{e^2 \tau n_e}{1-\text{i}\omega\tau} 
    \frac{1}{m_L},
\end{align}
sowie die Beiträge der verschiedenen optischen Phononfrequenzen 
($\nu_{LO}=\SI{292}{\per\centi\meter}$, $\nu_{TO}=\SI{268}{\per\centi\meter}$) 
sowie die Phononendämpfung ($\Gamma=\SI{2.5}{\per\centi\meter}$). Für die 
relative dielektrische Konstante wurde der Wert $\epsilon_{\infty}=10.37$ 
verwendet.

\begin{tabular}{l c c c}
    Probe & $\tau$ in $10^{-15}$s & $\eta$ in $10^{18} \frac{1}{\text{cm}}$ &
   $m$ in $m_e$ \\
   \hline
   $\circled{2}$ & $100\pm5$ & $0.9\pm0.1$ & $0.060\pm0.005$ \\
   $\circled{4}$ & $9.5\pm1.0$ & $1.10\pm0.05$ & $0.0055\pm0.0005$
\end{tabular}




% =============================================================================

\subsection{Zusammenfassung}

Während des Festkörperoptik-Experimentes konnten mit Hilfe eines 
FTIR-Spektrometer bequem verschiedene Halbleiterproben untersuchen. Zuerst 
konnten wir ein hochauflösendes Spektrum einer Globar-Lampe aufnehmen und 
innerhalb dieses Spektrum deutlich die Absorptionslinien von Kohlenstoffdioxid 
und Wasser nachweisen und im Fall von Kohlenstoffdioxid sogar einzelne 
Rotations-Schwingungsübergänge auflösen.
Anschließend wurde das Signal zu Rausch-Verhältnis untersucht und nachgewiesen, 
dass das Rauschen nicht mit $1/\sqrt{N}$ mit der Anzahl der Spiegeldurchläufe 
zusammenhängt.\\
Bei der Untersuchung von Galliumarsenid und Silizium-Proben wurden aus den 
Relfexions und Tranmissionsspektren aufgenommen. Durch Anfitten von 
theoretischen Daten konnte die dielektrische Funktion bestimmt und daraus der 
Brechungsindex berechnet werden.
Eine gesonderte Betrachtung der der Interbandanregung ermöglichte es die 
Bandlücke des  GaAs und des undotierten Si zu bestimmen. Wir erhielten für GaAs 
undotiert $\SI{1.384}{\eV}$ und für GaAs dotiert $\SI{1.36}{\eV}$. Für Silizium, 
dass eine indirekte Bandlücke besitzt, musste der Einfluss der Phononen 
angefittet werden. Wir errechneten die Bandlücke von undotiertem Silizium in 
Übereinstimmung zu $\SI{1.108}{\eV}$. Die Energie der an diesem Prozess 
beteiligen Phononen wurde auf $\SI{48}{\milli\eV}$ bestimmt. Der Beitrag der 
freien Ladungsträger wurde in Metallen und in Halbleitern diskutiert und die 
wesentlichen Unterschiede aufgezeigt. Als letztes konnten durch Anfitten der 
Reflektivitäten um die Plasmakante die Leitfähigkeitsmassen, die 
Stromrelaxationszeiten und die Dotierkonzentrationen bestimmt werden. Für 
Galliumarsenid bestimmten wir eine Dotierkonzentration von 
$n_e=\SI{0.9e18}{\per\centi\meter^3}$, eine Stomrelaxationszeit von 
$\tau=\SI{0.1}{\pico\second}$ und einer effektiven Elektronenmasse von 
$0.06m_e$. Die Dotierkonzentration der Siliziumprobe unterschied sich mit 
$n_e=\SI{1.1e18}{\per\centi\meter^3}$ nur wenig. Dagegen lagen die 
Stomrelaxationszeit von $\tau=\SI{0.0095}{\femto\second}$ und die effektive 
Masse von $0.0055m_e$ deutlich unterhalb der Werte in Galliumarsenid.\\
Insgesamt konnten die deutlichen Unterschiede der beiden Halbleiterarten sowie 
der Dotierkonzentrationen erkannt und deren Auswirkungen erklärt werden.
% =============================================================================
\begin{thebibliography}{}   
% =============================================================================

\bibitem{ansyco} http://www.ansyco.de/CMS

\bibitem{fkotheorie}Anleitung Festkörperoptik, Homepage Universität Würzburg

\bibitem{chriscolose} 
http://chriscolose.wordpress.com/2010/03/02/global-warming-mapsgraphs-2/

\bibitem{kittel} Kittel, C., Introduction to Solid State Physics, 6th Ed., New York:John Wiley, 1986, p. 185. 

\end{thebibliography}

\end{document}
