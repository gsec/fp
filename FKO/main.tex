
         %% %% %% %%         Preamble & Options          %% %% %% %%
%~~~~~~~~~~~~~~~~~~~~~~~~~~~~~~~~~~~~~~~~~~~~~~~~~~~~~~~~~~~~~~~~~~~~~~~~~~~~~~%
\documentclass[paper=a4,fontsize=10pt,DIV=18,twocolumn,parskip=half]{scrartcl}
%%%%%%%%%%%%                      PACKAGES                           %%%%%%%%%%%
%~~~~~~~~~~~~~~~~~~~~~~~~~~~~~~~~~~~~~~~~~~~~~~~~~~~~~~~~~~~~~~~~~~~~~~~~~~~~~~%
\usepackage[utf8]{inputenc}		% utf8 encoding für umlaute und sonderzeichen
\usepackage[ngerman,english]{babel} 	% hyphenations in  different languages
\usepackage[tbtags]{amsmath}    % tbtags sets the number of split 
                                % environments at the end 
\usepackage{amssymb} 	    	% symbols
\usepackage{amsfonts} 	    	% font
\usepackage{textcomp} 		    % degree signs -> \textdegree
\usepackage{siunitx}		    % correct units and spacing
\usepackage{hyperref}	        % references and links
\usepackage{graphicx}           % embedd images
\usepackage{tikz}			    % tikz painting, e.g circled text
\usepackage{subfig}				% subfigure for mutliple pictures
\usepackage{color}              % define own colors
\usepackage[font=small,labelfont=bf,format=plain,margin=10pt]{caption}
\usepackage[noabbrev]{cleveref}
\usepackage{bijan_commands}     % self defined commands and shorthand notations
                                % !! has to be added manually !!


%\usepackage{pdfsync}		    % pdf synchronization (reverse search)
%\usepackage{booktabs}	        % nice table lines
%\usepackage{fancyhdr} 			% headers and footers
%\usepackage{fancybox} 			% boxes with rounded corners 
%\usepackage{wrapfig} 		    % figures floating beside the text
%\usepackage{picinpar}			% textbox beside the text
%\usepackage{placeins}			% float barriers in sections
%\usepackage{moreverb}
%\usepackage{listings}
%\usepackage{pifont} 			% DING fonts (used in the acknowledgement)
%\usepackage{scrtime}           % current time 
%\usepackage[numbers,sort&compress]{natbib} 		
                                % sorted references  with clickable URL 
%\usepackage{makeidx} 			% index at the end
%\usepackage[chapter,numbib]{tocbibind}     % add several things to the TOC


%%%%%%%%%%%                        SETUP                             %%%%%%%%%%%
%~~~~~~~~~~~~~~~~~~~~~~~~~~~~~~~~~~~~~~~~~~~~~~~~~~~~~~~~~~~~~~~~~~~~~~~~~~~~~~%
% color:
\definecolor{darkblue}{rgb}{0,0,0.5}
\definecolor{lila}{rgb}{0.3,0,0.3}
\definecolor{turq}{rgb}{0,0.1,0.4}

% siunitx:
\sisetup	{separate-uncertainty, per-mode=fraction}

% hyperref:
\hypersetup{	pdftex,
                colorlinks=true,
                backref=page,
                linkcolor=darkblue,     % usual links
                filecolor=red,
                citecolor=turq,  	    % for bibliographic
                urlcolor=lila,  		% for Emails and URLs
                pdfpagelabels=true, 	% that the pagenumbering is ok 
                breaklinks=true,
                plainpages=false,
                bookmarks=true, 
                bookmarksnumbered=false
                %pdftitle={--title--},
                %pdfauthor={--author--},
                %pdfsubject={},
                %pdfkeywords={--kind-of-work--},
            }

% hyperref:
% set black for printing	
% \hypersetup{
    % colorlinks,
    % citecolor=black,
    % filecolor=black,
    % linkcolor=black,
    % urlcolor=black
    % }

%tocibind:
%\setcounter{tocdepth}{2} % change to 1 that not so deep (smaller TOC)
           % defines packages and setups
%~~~~~~~~~~~~~~~~~~~~~~~~~~~~~~~~~~~~~~~~~~~~~~~~~~~~~~~~~~~~~~~~~~~~~~~~~~~~~~%
\numberwithin{equation}{section}    % number equations after sections
\columnsep20pt                      % width between \twocolumns
\linespread{1.2}
\allowdisplaybreaks[1]              % permissiveness of page breaks in equations
                                    % 1 ="allow page breaks but avoid them" 
                                    % and 4="break whenever you want".

%%% Spacings:
\setlength{\headheight}{2.0\baselineskip}       
% Fixes the 'small headhight'

\renewcommand*{\chapterheadstartvskip}{\vspace{0\baselineskip}} 
% Spacing Pagehead-Headline. Standard: 2

\renewcommand*{\chapterheadendvskip}{\vspace{\baselineskip}}
% Spacing Headline-Text

%%%%%%%%% %%%%%%%%% %%%%%%%%% %%%%%%%%% %%%%%%%%% %%%%%%%%% %%%%%%%%% %%%%%%%%%%


\begin{document}

\title{Festkörperoptik}                  
\author{Guilherme Stein \& Ulrich Müller}         
\date{}                             % Turn off automatic date
\twocolumn[\begin{@twocolumnfalse}
\vspace{-3em}
\maketitle      
% =============================================================================
\begin{abstract}      
% =============================================================================
\vspace{-2em}
    \noindent {\small FKO abstact ... }
\vspace{1em}

  \noindent Betreuer: Dr. Christoph Brüne
  \hfill Experimentdurchführung: $10$. October 2013
  \begin{flushright}
    % Protokollabgabe: 12. Oktober 2012        
  \end{flushright}
\end{abstract}

\vspace{2em}
\end{@twocolumnfalse}
]


% =============================================================================

\section{Einleitung}

Halbleiter spielen heutzutage eine wichtige Rolle in vielen technischen 
Anwendungen. Die Bandlücke, ein Energiebereich in dem es keine 
quantenmechanischen Zustände gibt, macht diese Materialklasse interessant für 
viele Anwendungen in der Elektronik und der Optoelektronik. Silizium zählt dabei 
zu den indirekten Halbleitern und ist ein wichtiger Bestandteil vieler 
Metall-Oxid-Halbleiter Bauelemente. Für optisch aktive Halbleiterbauelemente 
werden in der Regel Halbleiter mit einem direkten Bandübergang verwendet, um 
eine hohe Wechselwirkung der Elektronen mit Licht zu ermöglichen. Genau diese 
Wechselwirkung ist für viele Bereiche der Physik von großem Interesse. \\
Ein etabliertes Werkzeug zur Charakterisierung von Halbleitern ist das 
Fourier-Transform-Infrarotspektrometers (FTIR). In diesem können Reflexions- und 
Transmissionsspektren einer Probe mit hoher Auflösung analysiert werden. Daraus 
lassen sich anschließend die Materialeigenschaften der Probe, wie komplexer 
Brechungsindex, Dotierkonzentrationen und Bandlücken berechnen. 

% =============================================================================

\section{Theorie}
\begin{itemize}
\item 6. Bandlücke und Impulsmatrixelemente
\item 7. Bandlücke und charakteristische Phononenfrequenz für Si
\item 8. Beitrag der freien Ladungsträger zur Reflexion
\item 9. Leitfähigkeitsmassen, Stromrelaxationszeiten und Dotierkonzentrationen
\end{itemize}


\begin{itemize}
\item Zusammenhang Dielektrizitätskonstante und komplexer Brechungsindex
\item Impulsmatrixelement
\end{itemize}

In diesem Abschnitt werden die wichtigsten Zusammenhänge die für die spätere 
Betrachtung nötig sind vorstellen. Dabei orientieren wir uns, soweit nicht 
anders angegeben, an \cref{fko-theorie}.

\subsection{Dispersionsrelation und Brechungsindex}

Aus den Maxwellgleichungen lässt sich die Wellengleichung herleiten, welche für 
einen Ansatz ebener Wellen uns folgende Dispersionsrelation liefert:

\begin{align}
    q^2 = \frac{\omega^2}{c^2}\epsilon(\omega)
\end{align}
mit dem Wellenvektor $q$, der Frequenz $\omega$ und $c$, der 
Lichtgeschwindigkeit im Vakuum.
$\epsilon(\omega)=\epsilon'(\omega)+i\epsilon''(\omega)$ beschreibt dabei die 
komplexe Dielektrizitätsfunktion. 
Diese hängt mit dem komplexen Brechungsindex zusammen:

\begin{align}
    \sqrt{\epsilon}=n+i\kappa
    \epsilon'=n^2-\kappa^2
    \epsilon''=2n\kappa
\end{align}

\subsection{Reflexion und Transmission}
Trifft Licht auf die Grenzfläche zwischen zwei unterschiedlichen Materialien, so 
wird ein gewisser Anteil der Intensität reflektiert. Ein anderer Teil des 
Lichtes gelangt in die Probe und kann dort absorbiert werden. Licht was nicht 
reflektiert oder absorbiert wird, verlässt die Probe als transmittiertes Licht 
wieder. Für die drei Koeffizienten diese drei Prozesse gilt deshalb

\begin{align}
    R+A+T=1.
\end{align}

Dabei sind die Koeffizienten eindeutig durch den komplexen Brechungsindex der 
Materialien festgelegt. Luft besitzt einen Brechungsindex von annähernd $1$.
Die Fresnelschen Formeln, die den Übergang des Lichtes von einem Medium in ein 
anderes beschreiben, vereinfachen sich bei senkrechtem Einfall deutlich. Man 
erhält für den Reflexionskoeffizienten

\begin{align}
    R=\left(\frac{1-N_2}{1+N_2}\right)^2.
\end{align}

und für den Transmissionskoeffizienten 

\begin{align}
    T=\left(\frac{2}{1+N_2}\right)^2.
\end{align}

\begin{figure}
    \centering
    \includegraphics[width=0.6\columnwidth]{Bilder/Reflexion.jpg}
    \caption{Mehrfachreflexion und -transmission an einer planparallelen 
    Schicht}
    \label{multi}
\end{figure}

Für eine planparallele Schicht muss man die wiederholte Reflexion und 
Transmission berücksichtigen, wie in \cref{multi} illustriert und erhält dabei 
dabei folgenden Ausdruck:
\begin{align}
    T_v= & \frac{(1-\rho)^2 \exp(-d\beta \nu )}{1-\rho^2 \exp (-2d\beta\nu)} \\ 
    \vspace{2em}
    R_v= & \rho+\frac{\rho (1-\rho)^2 \exp (-2 d \beta (\nu ))}{1-\rho^2 \exp(-2 
    d \beta \nu )}\\
    \vspace{2em}
    \text{wobei } \beta= & 4 \pi  \nu  \left| \kappa \right|
    \text{ und }\rho= \frac{\kappa ^2+(n-1)^2}{\kappa ^2+(n+1)^2}
    \label{trans}
\end{align}


Für den Übergang von Luft in ein anderes Medium werden die Koeffizienten 

\subsection{Lorentz Oszillator \& Drude Modell}

Aus dem Lorentz-Modell eines Elektrons als Oszillator erhält man eine 
dielektrische Funktion, die sich, obwohl durch klassische Überlegungen 
begründet, dem quantenmeschanischem Modell sehr nahe kommt. Dieses wird durch 
die Übergangsfrequenzen $\omega_j$, den Oszillatorstärken $f_j$ sowie den 
inversen der Lebensdauer der Zustände $\Gamma_j$:

\begin{align}
    \epsilon(\omega) = 1 + \frac{e^2}{m\epsilon_0}\sum_j\frac{Nf_j}{\omega_j^2 
    - \omega^2 - i\Gamma_j\omega}
    \label{lorentz}
\end{align}

wobei die Summe aller Oszillatorstärken sich zu eins summiert.

In der Impulsdarstellung lassen sich diese als Matrixelemente des 
Impulsoperators darstellen:
\begin{align}
    f_{ij}=\frac{2}{m\hbar\omega_{ij}}\abs{p_{ij}}^2
\end{align}


Das Drude Modell betrachtet nun freie Ladungsträger mit verschwindenden 
Resonanzfrequenzen. Mit der sog. Plasmafrequenz 
$\omega_p=\frac{Ne^2}{m\epsilon_0}$ lässt \cref{lorentz} umformulieren zu:

\begin{align}
    \epsilon(\omega)=1-\frac{\omega_p\tau^2}{1+\omega^2\tau^2}+
    i\frac{\omega_p\tau}{\omega+\omega^3\tau^2}
    \label{drude}
\end{align}

% =============================================================================

\section{Experimenteller Aufbau}

Alle Reflexions und Transmissionsmessungen finden im einem  FTIR statt, das 
schematisch in \cref{ftir} dargestellt ist. Dabei wird das Licht, von einer 
Infrarot-Lichtquelle (Globar-Stab oder Wolfram-Lampe) durch ein Interferometer 
geleitet und interferiert in Abhängigkeit der Position des beweglichen Spiegels. 
Das Frequenzspektrum wird dadurch auf eine neue Funktion $I(L)$ abgebildet. 
Diese Funktion beschreibt die Abhängigkeit der detektierten Intensität von der 
Spiegelposition und ist mathematisch eine Fouriertransformation des 
Frequenzspektrums. Im optischen Strahlengang kann nun der Teil der Intensität 
gemessen werden, der reflektiert bzw. transmittiert wird. Nach der Messung führt 
das FTIR die inverse Fouriertransformation aus und man erhält wieder die  
Intensität in Abhängigkeit der Frequenz. Um den Einfluss der Probenreflexion und 
-transmission zu isolieren, wird zu jeder Messung eine Referenzmessung 
durchgeführt. Dies wird im Fall der Transmissionsmessung mit leerem Probenraum 
und im Fall der Reflexionsmessung mit einem Goldspiegel durchgeführt, der im 
untersuchten Frequenzbereich nahezu perfekt reflektiert.
Untersucht werden verschiedene Proben, die Gasabsorption an Luft und das 
Rauschen des Spektrometers.

Folgende Proben standen uns zur Verfügung:

\begin{tabular}{l | l c }
        Probe & Material & Schichtdicke   \\
        \hline
       \texttt{Sample \circled{1}} & GaAs undotiert & \SI{470}{\micro\meter} \\
       \texttt{Sample \circled{2}} & GaAs dotiert   & \SI{440}{\micro\meter} \\
       \texttt{Sample \circled{3}} & Si undotiert   & \SI{530}{\micro\meter} \\
       \texttt{Sample \circled{4}} & Si dotiert     & \SI{530}{\micro\meter} \\
    %\hline
    %lk  \\
    %& \SI{470}{\micro\meter} 
    %& \SI{440}{\micro\meter} 
    %& \SI{530}{\micro\meter} 
    %& \SI{530}{\micro\meter}  


\end{tabular}

\begin{figure}
\centering
    \includegraphics[width=0.4\textwidth]{Bilder/FTIR}
    \caption{Aufbau des FTIR. Hauptbestandteile sind Lichtquelle, 
    Interferometer, Probenhalter und Detektor.}
    \label{ftir}
\end{figure}

% =============================================================================

\section{Versuchsdurchführung}
\begin{itemize}
\item Versuchsdurchführung
\end{itemize}

% =============================================================================

\section{Auswertung}

\subsection{Gasabsorption in Luft}
Am Anfang soll die Gasabsorption der Umgebungsluft untersucht werden.  Dazu 
verwenden wir die Globarquelle, die ein kontinuierliches Spektrum von 
$\SI{500}{\per\centi\meter}$ bis zu $\SI{5000}{\per\centi\meter}$ aussendet.  
Dieses Spektrum wird auf dem Weg zum Detektor von der Umgebungsluft an manchen 
Stellen absorbiert, was uns die Spektralbereiche starker Gasabsorption 
identifizieren lässt.
In \cref{Luft1} ist das gesamte aufgenommene Spektrum gezeigt.  Wir können 
jeweils zwei Bereiche der Absorption von Wasser (\cref{Luft2},\cref{Luft3}) und 
Kohlenstoffdioxid (\cref{Luft4},\cref{Luft5}) zuordnen.  Deutlich zu erkennen sich 
dabei auch die im Fall von Kohlenstoffdioxid die P- und R-Zweige der 
Rotations-Schwingungs-Spektren.

\subsection{Signal Rausch Verhältnis}
Mit geringer Auflösung messen wir ohne eingebaute Probe das Spektrum der 
Globar-Quelle und normieren es auf eine zweite Messung mit gleichen 
Einstellungen. Diese Messung wiederholen wir mit $10$, $50$ und $100$ 
Spiegeldurchläufe (Scans) und untersuchen, wie das Rauschen mit zunehmender 
Anzahl an Scans abnimmt.\\
Zur Auswertung wählen wir einen Bereich in dem die Intensität der einzelnen 
Spektren ausreichend hoch ist. Das heißt einen Bereich ohne Absorptionslinien in 
der Nähe der maximalen Intensität ($\SI{1000}{\per\centi\meter}$ bis 
$\SI{2200}{\per\centi\meter}$). In diesem Bereich erhalten wir vom FTIR 
normierte Intensitätswerte, die um die $1$ schwanken. Aus diesen 
Intensitätswerten, bestimmen wir die Standardabweichung vom Mittelwert und 
interpretieren diese als Rauschen der Daten. Um zu untersuchen, ob das Rauschen 
einem $1/\sqrt{N}$-Zusammenhang folgt, plotten wir in \cref{snr} das Rauschen 
über $1/\sqrt{N}$. Den Fehler der Standardabweichungen erhalten wir mit 
Gaußscher Fehlerfortpflanzung aus den Fehlern der einzelnen normierten 
Intensitätswerten, für die wir wiederum die Größe des Rauschen annehmen.


\subsection{Brechzahl von GaAs und Si}
Die untersuchten Proben bestehen aus mehreren hundert Nanometer dicken 
Schichten.
Untersucht man das Reflexionsspektrum dieser Schichten, so kommt es zwischen den 
Grenzflächen dieser Schichten zur Interferenz, die sich als Modulation im 
Reflexionsspektrum bemerkbar machen.  Wir bestimmen an verschiedenen Stellen im 
Spektrum jeweils den Abstand von $50$ Interferenzmaxima. Die Ablesegenauigkeit 
der Peaks nehmen wir mit $\SI{\pm 1}{\centi\meter^{-1}}$ an.
Mit Hilfe der Probendicken (GaAs undotiert: $\SI{470}{\micro\meter}$, GaAs 
dotiert: $\SI{440}{\micro\meter}$, Si undotiert: $\SI{530}{\micro\meter}$) lässt 
sich nun der der Realteil des Brechungsindexes der verschiedenen Materialien an 
verschiedenen Stellen im Spektrum bestimmen.

\begin{tabular}{ l | c c c }
  Wellen- & n & n&n\\
  zahl & GaAs undotiert & GaAs dotiert & Si undotiert \\
  \hline
  500 & 3.452(32) & - & 3.343(34) \\
  1000 & 3.245(28) & - & 3.348(34) \\
  1500 & 3.243(28) & - & 3.369(34) \\
  2000 & 3.224(28) & 3.303(27) & 3.363(34) \\
  2500 & 3.243(28) & 3.303(27) & 3.370(34) \\
  3000 & 3.263(28) & 3.284(27) & 3.394(35) \\
\end{tabular}

Die entsprechenden Dielektrizitätskonstanten ergeben sich aus dem Quadrat der 
komplexen Brechungsindizes, zu deren Berechnung uns aber noch die Auswertung der 
Absorptionsmessung fehlt.

\subsection{Reflexion und Transmission der Proben}

\subsection{Brechzahlen, Extinktions- und Absorptionskoeffizienten}

Das OPUS Programm liefert uns eine Tabelle indem für die einzelnen Messungen das 
Spektrum als 2-Tupel von Intensität und Wellenzahl repräsentiert wird. Um einen größeren Frequenzbereich 
abzudecken, wurde bei der Messung jeweils ein Spektrum mit einem Globar-Stab und 
einer Wolfram Lampe durchgeführt. Aus den beiden Spektren der jeweiligen Probe, 
wird das Spektrum, welches zur weiteren Untersuchung benutzt wird, 
zusammengesetzt. Dabei wurden versucht, starke Oszillationen zu vermeiden. Diese 
treten auf, wenn die Intensität der Lampe in einem Frequenzbereich gering ist, 
und bei der Normierung das Rauschen einen erheblichen Beitrag zum Signal 
leistet. Auch wurde versucht die Stetigkeit des Spektrums zu erhalten. Um das 
Rauschen zu mininieren wurde eine Glättung durchgeführt und die resultierenden 
Punkte zu einer Funktion interpoliert. Diese wurden dann mit den theoretischen 
Werten aus \cref{trans} verglichen. Es wurde für die jeweilige Wellenzahl ein $n$ und 
$\kappa$ gefittet, welche die Differenz aus Theorie und experimentellen Daten 
verschwinden ließ. Somit erhält man $\kappa(\nu)$, $n(\nu)$ und $\beta(\nu)$ 
welche in \cref{brechzahlen} aufgetragen sind. 

\begin{figure}
    \centering
    \begin{subfigure}{\columnwidth}
        \includegraphics[width=\textwidth]{Bilder/brechzahl}
        \caption{Realteil vom Brechungsindex $n$}
        \label{bz}
    \end{subfigure}
    \begin{subfigure}{\columnwidth}
        \includegraphics[width=\textwidth]{Bilder/extinktion.pdf}
        \caption{Imaginärteil vom Brechungsindex $\kappa$}
        \label{ex}
    \end{subfigure}
    \begin{subfigure}{\columnwidth}
        \includegraphics[width=\textwidth]{Bilder/absorption.pdf}
        \caption{Absorptionskoeffizient $\beta$}
        \label{ab}
    \end{subfigure}
    \caption{Die verschiedenen Koeffizienten unserer untersuchten Proben.  
    \texttt{Sample 1} $\rightarrow$ \text{rot},
    \texttt{Sample 2} $\rightarrow$ \text{grün}, 
    \texttt{Sample 3} $\rightarrow$ \text{blau}}
    \label{brechzahlen}
\end{figure}

\subsection{Bandlücke und Impulsmatrixelemente}

Erhöht man die Energie, ausgehend von einer Wellenzahl unterhalb der Bandlücke, über die Energie der Bandlücke, so findet man, dass der Absorptionskoeffizient stark zunimmt. Der Grund dafür liegt in der Anzahl der Zustände, die oberhalb der Bandlücke einen Übergang ermöglichen. Im idealisierten Modell nimmt dabei die Zustandsdichte im dreidimensionalen System wurzelförmig zu und immer mehr Lorentzoszillatoren tragen zur Erhöhung des Absorptionskoeffizient bei.\\
Um die Bandlücke und das Impulsmatrixelement der Übergänge zu bestimmen berechnen wir zunächst aus \cref{??} den Imaginärteil $\epsilon^{``}$ des undotierten und dotierten GaAs. Nun tragen wir $(\epsilon^{``}\cdot \omega^2)^2$ über die Anregungswellenzahl inTabelle \ref{??} auf. Der theoretische Verlauf dieser Funktion oberhalb der Bandlücke entspricht dabei

\begin{align}
   (\epsilon^{``}\cdot \omega^2)^2=\frac{e^4 (2\mu)^3 |p_{cv}|^4}{\epsilon_0^2 m_e^4 \hbar^2} \cdot (\hbar \omega - E_g)
\end{align}

Wir plotten zu dem Messpunkten eine theoretische Kurve, die unserer Meinung nach die Steigung nach dem Knick am besten wiedergibt und die Messpunkte etwa am steilsten Punkt tangentiert. Zu dieser Kurve plotten wir zwei Hilfsfunktionen, die die Messdaten etwas weiter ober und unten tangentieren, um eine Abschätzung der Unsicherheit von Bandlücke und Impulsmatrixelement zu erhalten.
Die Ergebnisse sind in Tabelle \ref{ime} dargestellt.

\begin{table}
	\begin{center}
\begin{tabular}{ l | c c }
  Material & Bandlücke/ $\SI{}{\eV}$ & Impulsmatrixelement\\
  \hline
  GaAs undotiert & 1.384(1) & 1.73(1) \\
  GaAs dotiert & 1.3591(5) & 1.59(3)   
\end{tabular}
\caption{Ermittelte Werte für die Bandlücke von Galliumarsenid und entsprechende Impulsmatrixelement.}
  \label{ime}  
	\end{center}
\end{table}
Die von uns bestimmte Bandlücke von GaAs liegt mit $\SI{1.384}{\eV}$ um $\SI{50}{\milli\eV}$ unterhalb des Literaturwertes von 1.43 eV (bei $\SI{300}{\K}$). Der Grund liegt in der Aufweichung der Bandkante, dem sogenannten Urbachtail, der die Einflüsse von
Exzitonen, Phononen, Verunreinigungen und Defekte sowie die Dotierungen im Material berücksichtigt. 

\subsection{Bandlücke und charakteristische Phononenfrequenz für Silizium}

Bei Silizium handelt es sich um einen indirekten Halbleiter. Daher werden die Übergänge vom Valenz- zum Leitungsband maßgeblich von den Phononen im Material beeinflusst. Wir nutzen den Zusammenhang 
\begin{align}
\beta _{\text{em} }=\left(n_{\text{ph}}+1\right)\frac{\left(\hbar \omega -\hbar \omega _{\text{ph}}-E_g\right){}^2}{(\hbar \omega )^2}
\end{align}
und 
\begin{align}
\beta _{\text{abs} }=<n_{\text{ph}}>\frac{\left(\hbar \omega +\hbar \omega _{\text{ph}}-E_g\right){}^2}{(\hbar \omega )^2}
\end{align}
der den Beitrag der Phononenabsorption und \mbox{-emission} beschreibt. Die theoretische Absorption wird dann zusammen mit den experimentellen Daten, als $\sqrt{\beta _{\text{em} }+\beta _{\text{abs} }}\nu $ linearisiert, über der Wellenzahl in \cref{??} geplottet. Der theoretische Verlauf wurde dabei durch Variation der Bandlücke und der Phononenenergie möglichst gut den experimentellen Daten angepasst. Die Werte für den besten Fit sind in \cref{phe} aufgelistet.
\begin{table}
	\begin{center}
\begin{tabular}{ l | c c }
  Material & Bandlücke/$\SI{}{\eV}$ & Phononenenergie/$\SI{}{\milli\eV}$\\
  \hline
  Si undotiert & 1.108(3) & 48(2)
\end{tabular}
  \caption{Ermittelte Werte für die Bandlücke von undotiertem Silizium und der Phononenenergie der bei Übergang beteiligten Phononen.}
  \label{phe}
	\end{center}
\end{table}

Um den Fehler der ermittelten Werte abzuschätzen, haben wir wie in vorherigen Abschnitt die Parameter der Bandlücke und der Phononenenergie variiert und so einen Bereich ermittelt, in dem die theoretische Absorption noch gut das Experiment beschreibt.
Die Ermittelte Bandlücke von Silizium stimmt gut mit dem Literaturwert von $\SI{1.11}{\eV}$ \cite{kittel} (bei $\SI{300}{\K}$) überein.


\begin{itemize}
\item 8. Beitrag der freien Ladungsträger zur Reflexion
\item 9. Leitfähigkeitsmassen, Stromrelaxationszeiten und Dotierkonzentrationen
\end{itemize}

\subsection{Drude Modell und characteristische Parameter}

In den dotierten Proben wurde die Stromrelaxationszeit $\tau$, die 
Dotierkonzentration $\eta$ sowie die effektive Masse $m$ bestimmt. Die 
gemessenen Reflektionsspektren wurden an ein theoretisches Model mit 
entsprechenden freien Parametern angefittet. Der Fit berücksichtigt die 
Hochfrequenzleitfähigkeit nach Drude:

\begin{align}
    \sigma(\omega) = \frac{e^2 \tau n_e}{1-\text{i}\omega\tau} 
    \frac{1}{m_L},
\end{align}
sowie die Beiträge der verschiedenen optischen Phononfrequenzen 
($\nu_{LO}=\SI{292}{\per\centi\meter}$, $\nu_{TO}=\SI{268}{\per\centi\meter}$) 
sowie die Phononendämpfung ($\Gamma=\SI{2.5}{\per\centi\meter}$). Für die 
relative dielektrische Konstante wurde der Wert $\epsilon_{\infty}=10.37$ 
verwendet.

\begin{tabular}{l c c c}
    Probe & $\tau$ in $10^{-15}$s & $\eta$ in $10^{18} \frac{1}{\text{cm}}$ &
   $m$ in $m_e$ \\
   \hline
   $\circled{2}$ & $100\pm5$ & $0.9\pm0.1$ & $0.060\pm0.005$ \\
   $\circled{4}$ & $9.5\pm1.0$ & $1.10\pm0.05$ & $0.0055\pm0.0005$
\end{tabular}




% =============================================================================

\subsection{Zusammenfassung}
\begin{itemize}
\item Zusammenfassung
\end{itemize}

% =============================================================================
\begin{thebibliography}{}   
% =============================================================================

\bibitem{ansyco} http://www.ansyco.de/CMS

\bibitem{fko-theorie}Anleitung Festkörperoptik, Homepage Universität Würzburg

\bibitem{chriscolose} 
http://chriscolose.wordpress.com/2010/03/02/global-warming-mapsgraphs-2/

\bibitem{kittel} Kittel, C., Introduction to Solid State Physics, 6th Ed., New York:John Wiley, 1986, p. 185. 





\bibitem{murray} Murray, Lee M.; Yildirim, Asli; Provence, Sydney R.; Norton, 
Dennis T.; Boggess, Thomas F.; Prineas, John P. (2013): Causes and elimination 
of pyramidal defects in GaSb-based epitaxial layers. In: J. Vac. Sci. Technol. B 
31 (3), S. 03C108. DOI: 10.1116/1.4792515.
  
\bibitem{merikoski} J. Merikoski; S.C. Ying (1997): Collective diffusion on a 
stepped substrate. In: surface science letters.

\bibitem{nanotec} http://www.nanotec.es/products/wsxm/download.php. 
\end{thebibliography}

\end{document}
