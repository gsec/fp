
         %% %% %% %%         Preamble & Options          %% %% %% %%
%~~~~~~~~~~~~~~~~~~~~~~~~~~~~~~~~~~~~~~~~~~~~~~~~~~~~~~~~~~~~~~~~~~~~~~~~~~~~~~%
\documentclass[paper=a4,fontsize=10pt,DIV=18,twocolumn,parskip=half]{scrartcl}
%%%%%%%%%%%%                      PACKAGES                           %%%%%%%%%%%
%~~~~~~~~~~~~~~~~~~~~~~~~~~~~~~~~~~~~~~~~~~~~~~~~~~~~~~~~~~~~~~~~~~~~~~~~~~~~~~%
\usepackage[utf8]{inputenc}		% utf8 encoding für umlaute und sonderzeichen
\usepackage[ngerman,english]{babel} 	% hyphenations in  different languages
\usepackage[tbtags]{amsmath}    % tbtags sets the number of split 
                                % environments at the end 
\usepackage{amssymb} 	    	% symbols
\usepackage{amsfonts} 	    	% font
\usepackage{textcomp} 		    % degree signs -> \textdegree
\usepackage{siunitx}		    % correct units and spacing
\usepackage{hyperref}	        % references and links
\usepackage{graphicx}           % embedd images
\usepackage{tikz}			    % tikz painting, e.g circled text
\usepackage{subfig}				% subfigure for mutliple pictures
\usepackage{color}              % define own colors
\usepackage[font=small,labelfont=bf,format=plain,margin=10pt]{caption}
\usepackage[noabbrev]{cleveref}
\usepackage{bijan_commands}     % self defined commands and shorthand notations
                                % !! has to be added manually !!


%\usepackage{pdfsync}		    % pdf synchronization (reverse search)
%\usepackage{booktabs}	        % nice table lines
%\usepackage{fancyhdr} 			% headers and footers
%\usepackage{fancybox} 			% boxes with rounded corners 
%\usepackage{wrapfig} 		    % figures floating beside the text
%\usepackage{picinpar}			% textbox beside the text
%\usepackage{placeins}			% float barriers in sections
%\usepackage{moreverb}
%\usepackage{listings}
%\usepackage{pifont} 			% DING fonts (used in the acknowledgement)
%\usepackage{scrtime}           % current time 
%\usepackage[numbers,sort&compress]{natbib} 		
                                % sorted references  with clickable URL 
%\usepackage{makeidx} 			% index at the end
%\usepackage[chapter,numbib]{tocbibind}     % add several things to the TOC


%%%%%%%%%%%                        SETUP                             %%%%%%%%%%%
%~~~~~~~~~~~~~~~~~~~~~~~~~~~~~~~~~~~~~~~~~~~~~~~~~~~~~~~~~~~~~~~~~~~~~~~~~~~~~~%
% color:
\definecolor{darkblue}{rgb}{0,0,0.5}
\definecolor{lila}{rgb}{0.3,0,0.3}
\definecolor{turq}{rgb}{0,0.1,0.4}

% siunitx:
\sisetup	{separate-uncertainty, per-mode=fraction}

% hyperref:
\hypersetup{	pdftex,
                colorlinks=true,
                backref=page,
                linkcolor=darkblue,     % usual links
                filecolor=red,
                citecolor=turq,  	    % for bibliographic
                urlcolor=lila,  		% for Emails and URLs
                pdfpagelabels=true, 	% that the pagenumbering is ok 
                breaklinks=true,
                plainpages=false,
                bookmarks=true, 
                bookmarksnumbered=false
                %pdftitle={--title--},
                %pdfauthor={--author--},
                %pdfsubject={},
                %pdfkeywords={--kind-of-work--},
            }

% hyperref:
% set black for printing	
% \hypersetup{
    % colorlinks,
    % citecolor=black,
    % filecolor=black,
    % linkcolor=black,
    % urlcolor=black
    % }

%tocibind:
%\setcounter{tocdepth}{2} % change to 1 that not so deep (smaller TOC)
           % defines packages and setups
%~~~~~~~~~~~~~~~~~~~~~~~~~~~~~~~~~~~~~~~~~~~~~~~~~~~~~~~~~~~~~~~~~~~~~~~~~~~~~~%
\numberwithin{equation}{section}    % number equations after sections
\columnsep20pt                      % width between \twocolumns
\linespread{1.2}
\allowdisplaybreaks[1]              % permissiveness of page breaks in equations
                                    % 1 ="allow page breaks but avoid them" 
                                    % and 4="break whenever you want".

%%% Spacings:
\setlength{\headheight}{2.0\baselineskip}       
% Fixes the 'small headhight'

\renewcommand*{\chapterheadstartvskip}{\vspace{0\baselineskip}} 
% Spacing Pagehead-Headline. Standard: 2

\renewcommand*{\chapterheadendvskip}{\vspace{\baselineskip}}
% Spacing Headline-Text

%%%%%%%%% %%%%%%%%% %%%%%%%%% %%%%%%%%% %%%%%%%%% %%%%%%%%% %%%%%%%%% %%%%%%%%%%


\begin{document}

\title{Festkörperoptik}                  
\author{Guilherme Stein \& Ulrich Müller}         
\date{}                             % Turn off automatic date
\twocolumn[\begin{@twocolumnfalse}
\vspace{-3em}
\maketitle      
% =============================================================================
\begin{abstract}      
% =============================================================================
\vspace{-2em}
    \noindent {\small FKO abstact ... }
\vspace{1em}

  \noindent Betreuer: Dr. Christoph Brüne
  \hfill Experimentdurchführung: $10$. October 2013
  \begin{flushright}
    % Protokollabgabe: 12. Oktober 2012        
  \end{flushright}
\end{abstract}

\vspace{2em}
\end{@twocolumnfalse}
]


% =============================================================================

\section{Einleitung}
\begin{itemize}
\item Einleitung
\end{itemize}

% =============================================================================

\subsection{Theorie}
\begin{itemize}
\item Theorie
\end{itemize}

% =============================================================================

\subsection{Experimenteller Aufbau}
\begin{itemize}
\item Experimenteller Aufbau
\end{itemize}

% =============================================================================

\subsection{Versuchsdurchführung}
\begin{itemize}
\item Versuchsdurchführung
\end{itemize}

% =============================================================================

\subsection{Auswertung}
\begin{itemize}
\item Auswertung
\end{itemize}

% =============================================================================

\subsection{Zusammenfassung}
\begin{itemize}
\item Zusammenfassung
\end{itemize}

% =============================================================================
\begin{thebibliography}{}   
% =============================================================================

\bibitem{arafin} Shamsul Arafin (2012): Electrically-Pumped GaSb-Based 
Vertical-Cavity Surface-Emitting Lasers. München.

\bibitem{weih} Weih, Robert; Kamp, Martin; Höfling, Sven (2013): Interband 
cascade lasers with room temperature threshold current densities below 100 
A/cm2. In: Appl. Phys. Lett. 102 (23), S. 231123. DOI: 10.1063/1.4811133.

\bibitem{vineis} C.J. Vineis; C.A. Wang; K.F. Jensen (2001): In-situ reflectance 
monitoring of GaSb substrate oxide desorption 2001.

\bibitem{murray} Murray, Lee M.; Yildirim, Asli; Provence, Sydney R.; Norton, 
Dennis T.; Boggess, Thomas F.; Prineas, John P. (2013): Causes and elimination 
of pyramidal defects in GaSb-based epitaxial layers. In: J. Vac. Sci. Technol. B 
31 (3), S. 03C108. DOI: 10.1116/1.4792515.
  
\bibitem{merikoski} J. Merikoski; S.C. Ying (1997): Collective diffusion on a stepped substrate. In: surface science letters.

\bibitem{nanotec} http://www.nanotec.es/products/wsxm/download.php. 
\end{thebibliography}

\end{document}
