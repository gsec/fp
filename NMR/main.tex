\documentclass[paper=a4,
	fontsize=10pt,
	DIV=18,
	twocolumn,
	parskip=half
	]{scrartcl}
\usepackage[font=small,labelfont=bf,format=plain,margin=10pt]{caption}
\usepackage{bijan_koma}
\usepackage[ngerman]{babel} 


%%%%%%%%%%%%%%%%%%%%%% Settings for packages %%%%%%%%%%%%%%%%%%%%%%%%%%%%%%%%%%

\usepackage[range-phrase={\,\,bis\,\,}]{siunitx}  % Correct typesetting of units
\sisetup{       
  separate-uncertainty,
  per-mode=fraction
}

\colorlet{darkblue}{blue!70!black}
\hypersetup{
  colorlinks,
  citecolor=darkblue,
  filecolor=darkblue,
  linkcolor=darkblue,
  urlcolor=black
}

\crefformat{equation}{Glg.~(#2#1#3)}
\crefformat{section}{Abschnitt~#2#1#3}
\crefformat{figure}{Abb.~#2#1#3}
\crefformat{table}{Tab.~#2#1#3}
\crefformat{chapter}{Kapitel~#2#1#3}


\addto\captionsngerman{             % Changes Abbildung->Abb.,etc. in caption 
  \renewcommand{\figurename}{Abb.}
  \renewcommand{\tablename}{Tab.}
}

\numberwithin{equation}{section}    % Number equations after sections, e.g. (1.2)

%%%%%%%%%%%%%%%%%%%%%% Headings and seperation lines %%%%%%%%%%%%%%%%%%%%%%%%%%

\usepackage[automark,markuppercase]{scrpage2}     % AUTOMATIC HEADINGS
\pagestyle{scrheadings}                           % Apply userdefined settings
\setheadsepline{.5 pt}                            % Width of seperation line
\setkomafont{pagehead}{\normalfont}               % Use normalfont for heading
\cfoot{\thepage}                                  % Page numbering 

%%%%%%%%%%%%%%%%%%%%%% Spacings %%%%%%%%%%%%%%%%%%%%%%%%%%%%%%%%%%%%%%%%%%%%%%%

\columnsep20pt                                  % Width inbetween \twocolumns
%\onehalfspacing                                 % 1.5 line spacing
\linespread{1.2}

\setlength{\headheight}{2.0\baselineskip}       % Fixes the 'small headhight'

\renewcommand*{\chapterheadstartvskip}{\vspace{0\baselineskip}} 
% Spacing Pagehead-Headline. Standard: 2

\renewcommand*{\chapterheadendvskip}{\vspace{\baselineskip}}
% Spacing Headline-Text

% Spacing in math environments \,\;.. 
%\thinmuskip=3mu % default
%\medmuskip=4mu plus 2mu minus 4mu % default is 4 mu p2 m4
%\thickmuskip=5mu plus 5mu % default

\allowdisplaybreaks[1]  % optional argument denoting permissiveness of page breaks 
% in equations. 1 ="allow page breaks but avoid them" and 4="break whenever you want".


\usepackage{url}
\usepackage[numbers]{natbib}


%%%%%%%%%%%%%%%%%%%%%%%%%%%%%%%%%%%%%%%%%%%%%%%%%%%%%%%%%%%%%%%%%%%%%%%%%%%%%%%%

\begin{document}

\title{Kernspinresonanz}                  
\author{Daniel Friedrich \& Ulrich Müller}         
\date{}                                % Turn off automatic date
\twocolumn[\begin{@twocolumnfalse}
\vspace{-3em}
\maketitle      
%=============================================================================
\begin{abstract}      
%=============================================================================
  \vspace{-2em}
  \noindent {\small Mithilfe von drei Röntgenanoden sowie verschiedenen
    Streuobjekten konnten wir die theoretischen Werte der
$K_{\alpha}$- und $K_{\beta}$-Linie von Kupfer, Eisen und Molybdän
    bestätigen. Zudem war die
Feinstruktur von Eisen und Molybdän 
    im Spektrum erkennbar.  Über das Duane-Hunt-Gesetz haben wir Plancksche
    Wirkungsquantum zu
%$h = \SI{6.645+-0.059e-34}{\joule\second}$
    bestimmt. 
%{R_{H} = \SI{14.02+-0.76}{\eV}} anhand der $K_{\beta}$ Linien
    Anhand des Effekts der inelastischen Streuung von Photonen an Elektronen
    haben wir die Compton-Wellenlänge zu
%$\lambda_c=\SI{2.25+-0.43}{\pico\meter}$
    ermittelt. Schließlich haben wir zwei
Laue-Aufnahmen
    eines Materials gemacht, den Reflexen Miller-Indices zugeordnet und damit 
    die Diamandstruktur der Probe
    identifiziert haben.
    }
\end{abstract}

  \vspace{1em}

\centerline{Betreuer: Dr. Charles Gould \hfill
   Versuchsdurchführung am 18. Oktober 2013}
\centerline{\hfill  Protokollabgabe am ??. Oktober 2013}
 

\vspace{2em}
%
\end{@twocolumnfalse}
]
%
% =============================================================================
\section{Einleitung}
\label{Einleitung}
%
Befindet sich ein Atomkern mit einem nichtverschwindem Spin in einem Magnetfeld, so kann er elektromagnetische Strahlung absorbieren sowie emittieren. Dieser Effekt wird als Kernspinresonanz (eng.: nuclear magnetic resonanz NMR) bezeichnet.
Zurückzuführen ist der Effekt auf das magnetische Moment, das durch den Spin des Atomkerns hervorgerufen wird. Dieses magnetische Moment besitzt, je nach Orientierung in einem äußeren Magnetfeld, unterschiedlich viel Energie. 
Die Energieaufspaltung eines Spins im äußeren Magnetfeld wurde zuerst im Jahre 1896 von Pieter Zeeman an Elektronen in einem Atom und 40 Jahre später von Isidor Rabi an Atomkernen nachgewiesen [citation needed].\\
Kleinste Unterschiede im lokalen magnetischen Feld von Atomkernen werden in der Chemie eingesetzt um Informationen über den Bindungszustand von Atomen zu gewinnen. Die Unterscheidung von Materialien aufgrund der Kernspinresonanz ermöglicht in der Magnetresonanztomographie zerstörungsfrei Bilder von organischen Proben in Echtzeit aufzunehmen. [citation needed, vielleicht was von der Uni zitieren]


%
% =============================================================================
\section{Theorie}
\label{Theorie}
%
\label{theorie}
Die Theorie der Kernspinresonanz ist auf die Wechselwirkung zwischen dem magnetischen Moment des Atomkerns und dem äußeren Magnetfeld zurück zu führen.
Das magnetische Moment des Kerns wird dabei von dessen Spin verursacht und folgt der Beziehung
\begin{align}
\vec{\mu}=g \frac{\mu_N}{\hbar}\vec{s}
\end{align}
mit g dem Landé-Faktor, $\hbar$ dem planck'schen Wirkungsquantum und $\mu_N$ dem Kernmagneton. Für ein Proton entspricht dabei das Kernmagneton äquivalent zum Bohrschen Magneton $\mu_N=\frac{e \hbar}{2m_p}$ mit $2m_p$ der Protonenmasse. Der Landé-Faktor des Protons beträgt etwa 5,59.\\
Befindet sich das magnetische Moment nun in einem Magnetfeld, so besitzt es die potentielle Energie 
\begin{align}
E_M=-\vec{\mu} \cdot \vec{B}
\end{align}
und ist somit in seiner energetisch günstigsten Position, wenn es parallel zum äußeren Feld ausgerichtet ist.
Nettomagnetisierung
Temperaturverhalten
Bloch-Gleichung
Relaxation
...
QM vs Klassisch
%
% =============================================================================
\section{Experimenteller Aufbau}
% =============================================================================
%
\label{Experiment}
Die Experimente und die Erzeugung der dafür notwendigen Röntgenstrahlung findet
in einem Vollschutzröntgengerät der Firma PHYWE statt...

\subsection{Charakteristische Röntgenstrahlung von Kupfer}

%
% =============================================================================
\section{Auswertung}
% =============================================================================
%
\label{AuswertungDaniel}
% ~~~~~~~~~~~~~~~~~~~~~~~~~~~~~~~~~~~~~~~~~~~~~~~~~~~~~~~~~~~~~~~~~~~~~~~~~~~~~
\subsection{Charakteristische Röntgenstrahlung von Kupfer}
\label{01_Auswertung}
% ~~~~~~~~~~~~~~~~~~~~~~~~~~~~~~~~~~~~~~~~~~~~~~~~~~~~~~~~~~~~~~~~~~~~~~~~~~~~~


\label{AuswertungUli}
% ~~~~~~~~~~~~~~~~~~~~~~~~~~~~~~~~~~~~~~~~~~~~~~~~~~~~~~~~~~~~~~~~~~~~~~~~~~~~~
\subsection{Laue-Aufnahme}
\label{07_Auswertung}
% ~~~~~~~~~~~~~~~~~~~~~~~~~~~~~~~~~~~~~~~~~~~~~~~~~~~~~~~~~~~~~~~~~~~~~~~~~~~~~
%
% =============================================================================
\section{Zusammenfassung}
% =============================================================================
%
Wir konnten mit dem Versuch einen guten Einblick in die Röntgenspektroskopie
gewinnen. Die charakteristischen Linien von Eisen, Molybdän und Kupfer wurden mit
recht hoher Genaugikeit nachgewiesen, wobei der größte Abstand von unseren
Bestwerten zu den Theoriewerten 
\SI{0.65}{\percent} betrag. Im Rahmen der Fehler gab es keine Abweichung. 
 Das empirische Gesetz zwischen der Intensität der
charakteristischen Strahlung und der Spannung zeigt systematische Abweichungen
für Spannungen ab \SI{30}{\kilo\volt} und sollte eher als Faustregel verstanden
werden. Das Duane-Hunt-Gesetz hingegen konnte gut bestätigt werden und hat uns
erlaubt, das Plancksche Wirkungsquantum zu bestimmen. Das Moseley-Gesetz wurde
ausführlich diskutiert und hat gute Abschätzungen für die Rydberg-Konstante
ergeben. Allerdings ist die Auswertung der \emph{Abschirmkonstante} $\sigma(Z)$
nicht wirklich sinnvoll. Mit dem Compton-Effekt konnte eine überraschend gute
Bestimmung der Compton-Wellenlänge durchgeführt werden. Eine vollständige
Aufnahme des Transmissionsspektrums von Al im gesamten Wellenlängenbereich der
Kupferanode würde helfen zu verstehen, warum die Näherung eines linearen
Spektrums solch gute Ergebnisse liefert. Die Laue-Aufnahme hat insgesamt gut
funktioniert. Allerdings könnte man die Aufhängung der Dentalfilme zum Beispiel
mit einer optischen Bank o.Ä. verbessern. Dadurch wird ein zentrales 
Auftreffen garantiert. Die Auflösung 
der Filme ist gut, eine größere Fläche wäre zwar wünschenswert, ist für die
Auswertung aber nicht unbedingt notwendig. 
%
% =============================================================================
%\begin{thebibliography}{}   
%% =============================================================================
%%
%%
%  \bibitem{levitt} Levitt, Malcolm H. (2008): Spin dynamics. Basics of nuclear magnetic resonance. 2nd ed. Chichester, England, Hoboken, NJ: John Wiley \& Sons.
%    \bibitem{hanson} Lars G. Hanson (2008): Is Quantum Mechanics Necessary for Understanding Magnetic Resonance?
%    Danish Research Centre for Magnetic Resonance, Copenhagen University Hospital, Hvidovre, Denmark.
%
%\end{thebibliography}
%
%

\small
\bibliographystyle{dinat}
\nocite{*}
\bibliography{lit}
\normalsize


\onecolumn
\pagestyle{empty}
% 

%=============================================================================
\section{Anhang}
\label{Anhang}
% =============================================================================

\end{document}
