\label{theorie}
Die Theorie der Kernspinresonanz ist auf die Wechselwirkung zwischen dem magnetischen Moment des Atomkerns und dem äußeren Magnetfeld zurück zu führen.
Das magnetische Moment des Kerns wird dabei von dessen Spin verursacht und folgt der Beziehung
\begin{align}
\vec{\mu}=g \frac{\mu_N}{\hbar}\vec{s}
\end{align}
mit g dem Landé-Faktor, $\hbar$ dem planck'schen Wirkungsquantum und $\mu_N$ dem Kernmagneton. Für ein Proton entspricht dabei das Kernmagneton äquivalent zum Bohrschen Magneton $\mu_N=\frac{e \hbar}{2m_p}$ mit $2m_p$ der Protonenmasse. Der Landé-Faktor des Protons beträgt etwa 5,59.\\
Befindet sich das magnetische Moment nun in einem Magnetfeld, so besitzt es die potentielle Energie 
\begin{align}
E_M=-\vec{\mu} \cdot \vec{B}
\end{align}
und ist somit in seiner energetisch günstigsten Position, wenn es parallel zum äußeren Feld ausgerichtet ist.
Nettomagnetisierung
Temperaturverhalten
Bloch-Gleichung
Relaxation
...
QM vs Klassisch

