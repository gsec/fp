Befindet sich ein Atomkern mit einem nichtverschwindem Spin in einem Magnetfeld, so kann er elektromagnetische Strahlung absorbieren sowie emittieren. Dieser Effekt wird als Kernspinresonanz (eng.: nuclear magnetic resonanz NMR) bezeichnet.
Zurückzuführen ist der Effekt auf das magnetische Moment, das durch den Spin des Atomkerns hervorgerufen wird. Dieses magnetische Moment besitzt, je nach Orientierung in einem äußeren Magnetfeld, unterschiedlich viel Energie. 
Die Energieaufspaltung eines Spins im äußeren Magnetfeld wurde zuerst im Jahre 1896 von Pieter Zeeman an Elektronen in einem Atom und 40 Jahre später von Isidor Rabi an Atomkernen nachgewiesen [citation needed].\\
Kleinste Unterschiede im lokalen magnetischen Feld von Atomkernen werden in der Chemie eingesetzt um Informationen über den Bindungszustand von Atomen zu gewinnen. Die Unterscheidung von Materialien aufgrund der Kernspinresonanz ermöglicht in der Magnetresonanztomographie zerstörungsfrei Bilder von organischen Proben in Echtzeit aufzunehmen. [citation needed, vielleicht was von der Uni zitieren]
