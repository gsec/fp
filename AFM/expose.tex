The III-V semiconductor system based on GaSb is commonly used for optical semiconductor devices with wavelengths beyond $\SI{2.3}{\micro\meter}$ \cite{arafin}. In Würzburg especially the interband cascade lasers, which are grown by MBE on GaSb substrate, made significant progress during the last years \cite{weih}. In order to grow devices with high performance it is inevitable to use high quality substrates with a minimum of defects. Despite the use of epi-ready substrates the wafers suffer from native oxide like $\mathrm{Ga}_2 \mathrm{O}_3$ and $\mathrm{Sb}_2 \mathrm{O}_3$ \cite{vineis}. The growth of devices on top of this oxide would lead to non-monocrystal layers. To remove this oxide a commonly used technique in Würzburg is to heat the substrate to about $\SI{580}{\degree}$ for short time. At this temperature the most of the oxide desorbes from the surface  but leaving holes in the surface with the size in the order of  $\SI{10}{\nano\meter}$ \cite{murray}. Hereupon a $\SI{200}{\nano\meter}$ GaSb buffer layer is grown at $\SI{485}{\degree}$ to flatten the surface. This method has been established during the last years although it has never been investigated whether a different technique would lead to smoother surfaces. 
To characterize the smoothness of a surface one needs a proper definition of this physical property:\\
	unsere Definition\\
A Atomic Force Microscope (AFM) is the perfect instrument to characterize this smoothness of the wafers. \\
	AFM Beschreibung\\
As the AFM doesn't work in situ we have to produce and investigate the surface at each step of the growth process to understand the mechanisms of oxide desorbtion and flattening of the surface. We going to characterize the single steps of the standard process which are: an untreated GaSb Wafer, the Wafer after the oxide desortion and after $\SI{200}{\nano\meter}$ GaSb buffer. To vary this process we want to test two aspects: first the increase of the GaSb buffer's growth temperature up to $\SI{500}{\degree}$ and $\SI{515}{\degree}$ and second the growth of a $\SI{30}{\nano\meter}$ GaSb/AlAsSb superlattice directly after oxide desorbtion. Recent research showed that the growth of AlAsSb shutting down the step-flow growth mode which is dominant in the growth of GaSb layers and is not very successful in flattening bigger defects like defects in pyramidal shape. The growth of a superlattice is nevertheless necessary to maintain the electrical conductivity of the sample.\\
It would be helpful to understand how these defects are removed from the surface and how the process can be improved as theses pyramidal defects tend to grow bigger as the growth progresses. After the growth of structures with a thickness of several microns these defects can even be observed by a optical microscope with a magnification of $50$.
\begin{figure}[htb]
  \includegraphics[width=\linewidth]{Bilder/A2749_50_1}
  \caption{At the samples' surface after several micrometer growth small pyramidal defects are visible. The image was taken by an optical microscope at a magnification of 50.}
  \label{bandluecke}
\end{figure}



%Materialsystem GaSb
%Optische Elemente/Laser
%Wachstum ohne Defekte wünschenswert
%native Oxid on epi-ready wafer.
%Standardverfahren beschreiben
%	Oxiddesorption
%	Bufferwachstum
%Wollen Verfahren variieren und  mit AFM zeigen, wie sich die Oberflächenrauigkeit verändert.
%AFM-Einleitung
%	Entdeckung
%	Funktion
%	etc...
%Proben
%1. nur Oxiddesorption
%2. Standard
%4. höhere Wachstumstemperaturen
%5. noch höhere Wachstumstemperaturen
%7. mit AlAsSb Übergitter 
%8. höhere Wachstumstemperaturen und AlAsSb 			    Übergitter 
%	oder 300nm Buffer
%	oder längere Oxiddesorption
%
%	
%Rauigkeit definieren
%	Über Blöcke der Größe Lambda mitteln; Dann Oberflächenspannung
%Mikroskopbilder hinzufügen
%
%Zitate

