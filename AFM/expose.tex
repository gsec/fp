\label{motivation}
The III-V semiconductor system based on GaSb is commonly used for optical 
semiconductor devices with wavelengths beyond $\SI{2.3}{\micro\meter}$ 
\cite{arafin}. In Würzburg especially the interband cascade lasers, which are 
grown by MBE on GaSb substrate, made significant progress during the last years 
\cite{weih}. In order to grow devices with high performance it is inevitable to 
use high quality substrates with a minimum of defects. Despite the use of 
'epi-ready' substrates the wafers suffer from native oxide like $\mathrm{Ga}_2 
\mathrm{O}_3$ and $\mathrm{Sb}_2 \mathrm{O}_3$ \cite{vineis}. The growth of 
devices on top of this oxide would lead to non-monocrystal layers. To remove 
this oxide a commonly used technique in Würzburg is to heat the substrate to 
about $\SI{580}{\degree}$ for a short time. At this temperature the most of the 
oxide desorbes from the surface but leaving holes in the surface with about  
$\SI{10}{\nano\meter}$ in size \cite{murray}. Hereupon a $\SI{200}{\nano\meter}$ 
GaSb buffer layer is grown at $\SI{485}{\degree}$ to flatten the surface. \\  
This method has been established during the last years although it is not clear 
whether a different technique would lead to smoother surfaces. Therefore one of 
the goals of this experiment is to determine a method of reducing the roughness 
on the wafer we want to grow on optical structures. This is important on the 
behalf of the optical quality the device can operate at.

From an intuitive point of view it is clear what roughness is. However,  
quantifying roughness in a mathematical way is not trivial. Common definitions 
use the standard deviation of the surface's mean height as \mbox{$ S_a = 
\frac{1}{A} \iint\limits_{A}  \vert z \left( x,y \right)  \vert \mathrm dx 
\mathrm dy $} and are suitable in many applications. Problems arise when 
applying this definition to non-flat surfaces. A soup bowl which appears flat 
and smooth when we are very close to the surface, shows a curvature when we look 
at the object as a whole. This example shows, that roughness is not independent 
of the scale. In our scope roughness will be a way of measuring the quality of a 
surface with respect to certain properties of the material.
%For this purpose we need to define roughness as a way to quantify 
%the properties of the surface of interest.
%------- [??] find ich super! Vielleicht auch gleich schreiben, dass wir etwa 
%suchen, was von der Skala abhängt, auf der man die Rauigkeit betrachtet. Wir 
%könnten vielleicht auch noch was dazu schreiben, wie Bildkorrekturen sich auf 
%die Rauigkeit auswirken.
%Since absolute height differences can also be intoduced by a constant slope, 
%not scattering the light in the sense of a defect
%-------
We think that in general the concept of a surface energy is a better approach to 
address the optical properties arising from surface irregularities. The more the 
size of the surface differs from a perfect flat surface the rougher the surface 
is. We want to model the surface as polygons connecting the mean heights of a 
discrete lattice.  The lattice size represents the scaling parameter mentioned 
above. Here we make the arbitrary choice of the AFM's resolution which will be 
the scale on which we measure the roughness of the samples.


%One of the simplest models of surface energy or surface 
%tension  is to assign an atom an energy depending on how many sides it has a 
%neighbouring atom. The neighbour positions are those of the nearest neighbours 
%in the lattice of the material.  In a simple cubic lattice the energy of a free 
%atom would be six, the energy of a surface atom would be one.
%

%Large fluctuations have a small effect, already represented by the surface 
%energy. On the other hand very small fluctuations, much smaller than the 
%wavelength of the light, also have a small effect on the optical properties.  
%They have a big, but also unwanted contribution to the surface energy.

%% we need to define the size of the 'atoms' with a parameter $\lambda$.
%A natural choice is to set $\lambda$ to the wavelength of the light which the 
%device will interact with. All smaller fluctiations are then neglected and 
%averaged to a mean height with the quantization of the axis in $x$ and $y$ 
%direction to the length scale of $\lambda$.
%\begin{align}
%    \Upsilon &=  \sum_i \epsilon_i \hfill ~ \\
%    \text{where} \quad \epsilon_i &=  \sum_{r=x,y,z} \Bigl( r_{i-1} + r_{i+1} 
%    \Bigr) \\
%    \text{and}  \quad r_j &=    \begin{cases}
%                1 &\mbox{if site contains an atom } \\
%                0 &\mbox{else }
%            \end{cases}
%\end{align}
The Atomic Force Microscope (AFM) is the perfect instrument to characterize this 
roughness of the wafers as it determines the height of the surface very 
precisely. The expected differences in height on the surface is about 
$\SI{10}{\nano\meter}$ which is within the resolution of the AFM.
As the AFM doesn't work in situ we have to produce and investigate the surface 
at each step of the growth process to understand the mechanisms of oxide 
desorbtion and flattening of the surface. We are going to characterize the 
single steps of the standard process which are: an untreated GaSb wafer, the 
wafer after the oxide desorbtion and after the growth of $\SI{200}{\nano\meter}$ 
GaSb buffer. To vary this process we want to test two aspects: first the 
increase of the GaSb buffer's growth temperature up to $\SI{500}{\degree}$ and 
$\SI{515}{\degree}$ and second the growth of a $\SI{30}{\nano\meter}$ 
GaSb/AlAsSb superlattice directly after oxide desorbtion. Recent research showed 
that the growth of AlAsSb shutting down the step-flow growth mode. This  
step-flow growth mode is dominant during the growth of GaSb layers and is not 
very successful in flattening bigger defects like defects in pyramidal shape 
\cite{murray}. The growth of a superlattice instead of a bulk layer is 
nevertheless necessary to maintain the electrical conductivity of the sample.
It would be helpful to understand how these defects can be removed from the 
surface and how the process can be improved. If the smoothing is not successful 
theses pyramidal defects tend to grow bigger as the growth progresses. After the 
growth of structures with a thickness of several microns these defects can even 
be observed by a optical microscope as shown in \cref{pyramidaldefects}.\\
\begin{figure}
  \includegraphics[width=\linewidth]{Bilder/A2749_50_1.jpg}
  \caption{At the sample's surface after several micrometer growth small 
  pyramidal defects are visible. The image was taken by an optical microscope at 
  a magnification of 50.}
  \label{pyramidaldefects}
\end{figure}
After sample production exposure to air can not be avoided. To reduce surface 
corrosion the samples will be produced tight before the experiment and stored in 
a nitrogen-flooded cabinet.
%After producing the samples we can not avoid an exposure to air.  Nevertheless 
%we are going to produce the samples right before the AFM experiment and store 
%them in a cabinet flooded by pure nitrogen in order to reduce surface corrosion.



%Materialsystem GaSb
%Optische Elemente/Laser
%Wachstum ohne Defekte wünschenswert
%native Oxid on epi-ready wafer.
%Standardverfahren beschreiben
%	Oxiddesorption
%	Bufferwachstum
%Wollen Verfahren variieren und  mit AFM zeigen, wie sich die 
%Oberflächenrauigkeit verändert.
%AFM-Einleitung
%	Entdeckung
%	Funktion
%	etc...
%Proben
%1. nur Oxiddesorption
%2. Standard
%4. höhere Wachstumstemperaturen
%5. noch höhere Wachstumstemperaturen
%7. mit AlAsSb Übergitter 8. höhere Wachstumstemperaturen und AlAsSb 			    
%Übergitter oder 300nm Buffer
%	oder längere Oxiddesorption
%
%	
%Rauigkeit definieren
%	Über Blöcke der Größe Lambda mitteln; Dann Oberflächenspannung
%Mikroskopbilder hinzufügen
%
%Zitate

