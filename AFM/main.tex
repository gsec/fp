\documentclass[paper=a4,fontsize=10pt,DIV=18,twocolumn,parskip=half]{scrartcl}
\usepackage[font=small,labelfont=bf,format=plain,margin=10pt]{caption}
\usepackage{bijan_koma}
\usepackage[ngerman]{babel} 


%%%%%%%%%%%%%%%%%%%%%% Settings for packages %%%%%%%%%%%%%%%%%%%%%%%%%%%%%%%%%%

\usepackage[range-phrase={\,\,bis\,\,}]{siunitx}  % Correct typesetting of units
\sisetup{       
  separate-uncertainty,
  per-mode=fraction
}

\colorlet{darkblue}{blue!70!black}
\hypersetup{
  colorlinks,
  citecolor=darkblue,
  filecolor=darkblue,
  linkcolor=darkblue,
  urlcolor=black
}

\crefformat{equation}{Glg.~(#2#1#3)}
\crefformat{section}{Abschnitt~#2#1#3}
\crefformat{figure}{Abb.~#2#1#3}
\crefformat{table}{Tab.~#2#1#3}
\crefformat{chapter}{Kapitel~#2#1#3}


\addto\captionsngerman{             % Changes Abbildung->Abb.,etc. in caption 
  \renewcommand{\figurename}{Abb.}
  \renewcommand{\tablename}{Tab.}
}

\numberwithin{equation}{section}    % Number equations after sections, e.g. (1.2)

%%%%%%%%%%%%%%%%%%%%%% Headings and seperation lines %%%%%%%%%%%%%%%%%%%%%%%%%%

\usepackage[automark,markuppercase]{scrpage2}     % AUTOMATIC HEADINGS
\pagestyle{scrheadings}                           % Apply userdefined settings
\setheadsepline{.5 pt}                            % Width of seperation line
\setkomafont{pagehead}{\normalfont}               % Use normalfont for heading
\cfoot{\thepage}                                  % Page numbering 

%%%%%%%%%%%%%%%%%%%%%% Spacings %%%%%%%%%%%%%%%%%%%%%%%%%%%%%%%%%%%%%%%%%%%%%%%

\columnsep20pt                                  % Width inbetween \twocolumns
%\onehalfspacing                                 % 1.5 line spacing
\linespread{1.2}

\setlength{\headheight}{2.0\baselineskip}       % Fixes the 'small headhight'

\renewcommand*{\chapterheadstartvskip}{\vspace{0\baselineskip}} 
% Spacing Pagehead-Headline. Standard: 2

\renewcommand*{\chapterheadendvskip}{\vspace{\baselineskip}}
% Spacing Headline-Text

% Spacing in math environments \,\;.. 
%\thinmuskip=3mu % default
%\medmuskip=4mu plus 2mu minus 4mu % default is 4 mu p2 m4
%\thickmuskip=5mu plus 5mu % default

\allowdisplaybreaks[1]  % optional argument denoting permissiveness of page breaks 
% in equations. 1 ="allow page breaks but avoid them" and 4="break whenever you want".

%%%%%%%%%%%%%%%%%%%%%%%%%%%%%%%%%%%%%%%%%%%%%%%%%%%%%%%%%%%%%%%%%%%%%%%%%%%%%%%%


\begin{document}

\title{Rasterkraftmikroskopie}                  
\author{Guilherme Stein \& Ulrich Müller}         
\date{}                                % Turn off automatic date
\twocolumn[\begin{@twocolumnfalse}
\vspace{-3em}
\maketitle      
% =============================================================================
\begin{abstract}      
% =============================================================================
  \vspace{-2em}
  \noindent {\small 
  % abstact 
    }
  \vspace{1em}

  \noindent Betreuer: Dr. Paolo Sessi
  \hfill Versuchsdurchführung: 04. Oktober 2012
  \begin{flushright}
    % Protokollabgabe: 12. Oktober 2012        
  \end{flushright}
\end{abstract}
\vspace{2em}
%
\end{@twocolumnfalse}
]
% =============================================================================
\addtocounter{section}{-1}
\section{Expose}
% =============================================================================
%
\input{Expose}
\label{Expose}
%
%% =============================================================================
%\section{Einleitung}
%% =============================================================================
%%
%\input{Einleitung}
%\label{Einleitung}
%%
%% =============================================================================
%\section{Theorie}
%% =============================================================================
%%
%\label{theorie}
Bewegen sich freie Elektronen in einem Magnetfeld, so wirkt auf sie die Lorentzkraft, die sie senkrecht zu ihrer Bewegungsrichtung ablenkt. Sind die Elektronen hingegen auf das Volumen eines Leiters limitiert, so baut sich, im stationären Fall bei kleinen B-Feldern, ein elektrisches Feld senkrecht zum Magnetfeld und der Bewegung der Elektronen auf, das die Lorentzkraft kompensiert. Die Spannung zwischen den Flanken des Leiters wird Hall-Spannung genannt. Der Hall-Widerstand ist nicht als klassischer Widerstand zu verstehen, sondern berechnet sich aus dem Verhältnis der Hall-Spannung und der verwendeten Stromstärke zu
\begin{equation}
R_{Hall}=\frac{U_y}{I}=\frac{B}{n_s e}.
\label{rhall}
\end{equation}
Mit der Flächenladungsträgerdichte $n_s$ im Material. Der Hall-Widerstand ist damit proportional zum angelegten Magnetfeld. \\
Verwendet man ein Material mit hohen Beweglichkeiten der Elektronen, kühlt die Probe auf tiefe Temperaturen und erhöht die Magnetfeldstärke, so findet man bei charakteristischen Werten des Hall-Widerstands Plateaus die von den Materialeigenschaften, der Magnetfeldstärke und der verwendeten Temperatur unabhängig zu sein scheinen.
Diese Plateaus treten in regelmäßigen Abständen auf und können durch den Zusammenhang 
\begin{equation}
R_{Hall}=\frac{h}{e^2} \frac{1}{i}
\label{rqhe}
\end{equation}
beschrieben werden. mit den ganzen Zahlen $i=1,2,...$ Das höchste messbare Plateau befindet sich demnach bei 
\begin{equation}
R_K=\frac{h}{e^2}= \SI{25812.807}{\ohm}
\end{equation}
und wird als Klitzing-Konstante für die Definition des Ohm verwendet.\\
Vergleicht man den Hall-Widerstand des klassischen Hall-Effekt \ref{rhall} mit dem des Quanten-Hall-Effekts \ref{rqhe}, so scheinen die beim Ladungstransport beteiligten Elektronen pro Volumen immer in Paketen zu $\frac{e B}{\hbar}$ zur Leistung beizutragen. Dieser Ausdruck wird als Entartungsgrad diskreter Landauniveaus bezeichnet die sich aus der vollen quantenmechanischen Beschreibung ergeben.
Dazu wird der Hamiltonoperatur für Elektronen im äußeren Magnetfeld aufgestellt:
\begin{equation}
H=\frac{1}{2 m} (\vec{p} - \frac{e \vec{A}}{i \hbar})^2.
\end{equation}
Lässt man nur Bewegungen der Elektronen in der x-y-Ebene zu und eicht das Vektorpotential auf  $\vec{A}= (0, \; B x, \; 0)$, so kann man mit dem Ansatz $\Phi(x,y)=C \cdot e^{i k_x k} u(x)$ die Schrödingergleichung auf eine Differenzialgleichung des harmonischen Oszillators bringen.
Die Energieeigenwerte beschreiben die Landauniveaus und besitzen die Form
\begin{equation}
E_n=\hbar \omega_c (n+\frac{1}{2})
\end{equation}
mit der Zyklotronfrequenz $\omega_c$. Diese Niveaus besitzen den Entartungsgrad
\begin{equation}
N_{LL}= \frac{e B}{\hbar}
\end{equation}
pro Volumen. ??ICH HAB MICH EIN BISSCHEN UM DIE ERKLÄRUNG GEDRÜCKT; WARUM ES DANN ÜBERHAUPT ZU EINER HALL SPANNUNG KOMMT: WENN MAN DEN QM-ANSATZ OHNE SPANNUNG MACHT, KOMMT HALT AUCH KEINE HALL-SPANNUNG RAUS?? \\
Gleichzeitig zu den Plateaus im Hallwiderstand, tritt ein anderer bemerkenswerter Effekt auf, der als Schubnikow-de-Haas-Effekt bezeichnet wird. Dabei verliert der Festkörper seinen Widerstand in Längsrichtung, so dass der angelegten Strom ohne Verluste im Festkörper transportiert wird. Liegt die Fermienergie zwischen zwei Landauniveaus, so gibt es für die Elektronen keine Zustände innerhalb der thermischen Energie in die gestreut werden kann. Die Leitung wird damit dissipationslos. Das Randkanalmodell beschreibt anschaulich, wie am Rand der Probe die Probenoberfläche die Energieniveaus der Landaulevel anhebt und so einen eindimensionalen leitenden Randkanal formt, indem ebenfalls dissipationslose Leitung möglich ist.
Das Verhalten der Probe wird also im wesentlich von der Magnetfeldstärke beeinflusst, die den Entartungsgrad der Landauniveuas erhöht und damit sowohl die Lage der Fermienergie als auch den Füllfaktor, d.h. die Anzahl der besetzten Landauniveaus, festlegt. Der Füllfaktor ist als
\begin{equation}
\nu=\frac{N_s}{N_{LL}}
\end{equation}
definiert, wobei $N_s$ der Teilchendichte und $N_l$ dem Entartungsgrad der Landauniveaus entspricht. Bei einem ganzzahligen Füllfaktor befindet man sich im Bereich eines Plateaus des Quanten-Hall-Effektes und man misst den beschriebenen Schubnikow-de-Haas-Effekt.\\
Idealer Weise sind diese Energieniveaus delta-förmig, in der Realität aber oft durch Streuung mit Phononen und Störstellen, sowie die Zeeman-Aufspaltung verbreitert. Damit sind die Bedingungen zu verstehen, bei denen der Quanten-Hall-Effekt auftritt. Bei starken Magnetfeldern treten wenige stark besetzte Landaulevel auf. Zudem verringern hohe Beweglichkeiten der Elektronen die Stoßverbreiterung. Als letztes werden hinreichend tiefe Temperaturen benötigt, damit nur Landauniveaus unterhalb der Fermienergie besetzt sind.\\

%Entdeckung des QHE Klaus von Klitzing 1980
%2DEG mit hohen beweglichkeiten
%Landau Eichung
%quantisierung erst bei (s k B T) hohen Feldstärken und niedrigen Temperaturen
%"Man
%würde daher erwarten, dass es nur zu einem geringen
%Ladungstransport kommt, und der Widerstand der Pro-
%be entsprechend hoch ist. Tatsächlich beobachtet man
%aber einen entgegengesetzten Effekt"
%\label{Theorie}
%%
%% =============================================================================
%\section{Experimenteller Aufbau}
%% =============================================================================
%%
%\input{Experiment}
%\label{Experiment}
%%
%% =============================================================================
%\section{Auswertung}
%% =============================================================================
%%
%
%\input{AuswertungGui}
%\input{AuswertungGui}
%
%\input{AuswertungUli}
%\input{AuswertungUli}
%%
%% =============================================================================
%\section{Zusammenfassung}
%% =============================================================================
%%
%Wir konnten mit dem Versuch einen guten Einblick in die Röntgenspektroskopie
gewinnen. Die charakteristischen Linien von Eisen, Molybdän und Kupfer wurden mit
recht hoher Genaugikeit nachgewiesen, wobei der größte Abstand von unseren
Bestwerten zu den Theoriewerten 
\SI{0.65}{\percent} betrag. Im Rahmen der Fehler gab es keine Abweichung. 
 Das empirische Gesetz zwischen der Intensität der
charakteristischen Strahlung und der Spannung zeigt systematische Abweichungen
für Spannungen ab \SI{30}{\kilo\volt} und sollte eher als Faustregel verstanden
werden. Das Duane-Hunt-Gesetz hingegen konnte gut bestätigt werden und hat uns
erlaubt, das Plancksche Wirkungsquantum zu bestimmen. Das Moseley-Gesetz wurde
ausführlich diskutiert und hat gute Abschätzungen für die Rydberg-Konstante
ergeben. Allerdings ist die Auswertung der \emph{Abschirmkonstante} $\sigma(Z)$
nicht wirklich sinnvoll. Mit dem Compton-Effekt konnte eine überraschend gute
Bestimmung der Compton-Wellenlänge durchgeführt werden. Eine vollständige
Aufnahme des Transmissionsspektrums von Al im gesamten Wellenlängenbereich der
Kupferanode würde helfen zu verstehen, warum die Näherung eines linearen
Spektrums solch gute Ergebnisse liefert. Die Laue-Aufnahme hat insgesamt gut
funktioniert. Allerdings könnte man die Aufhängung der Dentalfilme zum Beispiel
mit einer optischen Bank o.Ä. verbessern. Dadurch wird ein zentrales 
Auftreffen garantiert. Die Auflösung 
der Filme ist gut, eine größere Fläche wäre zwar wünschenswert, ist für die
Auswertung aber nicht unbedingt notwendig. 

%%
% =============================================================================
\begin{thebibliography}{}   
% =============================================================================
%

\bibitem{arafin} Shamsul Arafin (2012): Electrically-Pumped GaSb-Based Vertical-Cavity Surface-Emitting Lasers. München.

\bibitem{weih} Weih, Robert; Kamp, Martin; Höfling, Sven (2013): Interband cascade lasers with room temperature threshold current densities below 100 A/cm2. In: Appl. Phys. Lett. 102 (23), S. 231123. DOI: 10.1063/1.4811133.


\bibitem{vineis} C.J. Vineis; C.A. Wang; K.F. Jensen (2001): In-situ reflectance monitoring of GaSb substrate oxide desorption 2001.

\bibitem{murray} Murray, Lee M.; Yildirim, Asli; Provence, Sydney R.; Norton, Dennis T.; Boggess, Thomas F.; Prineas, John P. (2013): Causes and elimination of pyramidal defects in GaSb-based epitaxial layers. In: J. Vac. Sci. Technol. B 31 (3), S. 03C108. DOI: 10.1116/1.4792515.
  

\end{thebibliography}
%
%
\onecolumn
\pagestyle{empty}
%% =============================================================================
%\section{Anhang}
%\label{Anhang}
%% =============================================================================

\end{document}
