\section{Roughness of the surface}

One of the goals of this experiment, is to determine a method of reducing the 
roughness on the wafer we want to grow on optical structures. This is important 
on the behalf of ithe optical quality the device can operate at.
For this purpose we need to define roughness as a way to quantify the properties 
of the surface of interest.
%-------
%Since absolute height differences can also be intoduced by a constant slope, 
%not scattering the light in the sense of a defect
%-------
We think that the concept of a surface energy is a good approach to address 
those optical qualities. One of the simplest models of surface energy or surface 
tension  is to assign an atom an energy depending on how many sides it has a 
neighbouring atom. The neighbour positions are those of the nearest neighbours 
in the lattice of the material.  In a simple cubic lattice the energy of a free 
atom would be six, the energy of a surface atom would be one.
%
Large fluctuations have a small effect, already represented by the surface 
energy. On the other hand very small fluctuations, much smaller than the 
wavelength of the light, also have a small effect on the optical properties.  
Since they would have a big, but also unwanted contribution to the surface 
energy, we need to define the size of the 'atoms' with a parameter $\lambda$.
A natural choice is to set $\lambda$ to the wavelength of the light which the 
device will interact with. All smaller fluctiations are then neglected and 
averaged to a mean height with the quantization of the axis in $x$ and $y$ 
direction to the length scale of $\lambda$.
\begin{align}
    \rho = \sum_i \epsilon_i \\
    \text{where} \epsilon_i = \sum_{r=x,y,z}\sum_{j=i-1,i+1}r_j \\
    \text{and}  \\
    r_j =   \begin{cases}
                1 \mbox{if  site contains an atom }//
                0 \mbox{else }
            \end{cases}
\end{align}
