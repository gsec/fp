\label{theorie}
Bewegen sich freie Elektronen in einem Magnetfeld, so wirkt auf sie die Lorentzkraft, die sie senkrecht zu ihrer Bewegungsrichtung ablenkt. Sind die Elektronen hingegen auf das Volumen eines Leiters limitiert, so baut sich, im stationären Fall bei kleinen B-Feldern, ein elektrisches Feld senkrecht zum Magnetfeld und der Bewegung der Elektronen auf, das die Lorentzkraft kompensiert. Die Spannung zwischen den Flanken des Leiters wird Hall-Spannung genannt. Der Hall-Widerstand ist nicht als klassischer Widerstand zu verstehen, sondern berechnet sich aus dem Verhältnis der Hall-Spannung und der verwendeten Stromstärke zu
\begin{equation}
R_{Hall}=\frac{U_y}{I}=\frac{B}{n_s e}
\end{equation}
beschrieben. Mit der Flächenladungsträgerdichte $n_s$ im Material im Material. Der Hall-Widerstand ist damit proportional zum angelegten Magnetfeld. \\
Verwendet man ein Material mit hohen Beweglichkeiten der Elektronen, kühlt die Probe auf tiefe Temperaturen und erhöht die Magnetfeldstärke, so findet man bei charakteristischen Werten des Hall-Widerstands Plateaus die von den Materialeigenschaften, der Magnetfeldstärke und der verwendeten Temperatur unabhängig zu sein scheinen.
Diese Plateaus treten in regelmäßigen Abständen auf und können durch den Zusammenhang 
\begin{equation}
R_{Hall}=\frac{h}{e^2} \frac{1}{i}
\end{equation}
beschrieben werden. mit den ganzen Zahlen $i=1,2,...$ Das höchste messbare Plateau befindet sich demnach bei 
\begin{equation}
R_K=\frac{h}{e^2}= \SI{25812.807}{\ohm}
\end{equation}
und wird als Klitzing-Konstante bezeichnet.

Hall effekt
hohe Beweglichkeit, starke Magnetfelder und niedrige Temperaturen -> Plateaus
Klitzing konstante
Schubnikov-de-Haas Oszillationen
Quantenmechanische Beschreibung und Teile der Herleitung
Landau niveaus (Bedingungen erklären) und Ferminiveaus 
Randkanalmodell und Füllfaktor
Zeeman-Splitting


%Entdeckung des QHE Klaus von Klitzing 1980
%2DEG mit hohen beweglichkeiten
%Landau Eichung
%quantisierung erst bei (s k B T) hohen Feldstärken und niedrigen Temperaturen
%"Man
%würde daher erwarten, dass es nur zu einem geringen
%Ladungstransport kommt, und der Widerstand der Pro-
%be entsprechend hoch ist. Tatsächlich beobachtet man
%aber einen entgegengesetzten Effekt"