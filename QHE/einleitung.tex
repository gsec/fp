Über hundert Jahre nach Entdeckung des Hall-Effekts gelang es Klaus von Klitzing im Jahr 1980 die bei einem Hall-Experiment aufgetretenen Plateaus im Hall-Widerstand mit Hilfe der Plankkonstante und die Elementarladung zu deuten. Die Deutung durch die Plankkonstante machte deutlich dass es sich bei dem Effekt um einen makroskopischen Quanteneffekt handelt. Die Bildung von Plateaus im Hall-Widerstand bei wachsendem Magnetfeld wird seit dem 'Quanten-Hall-Effekt' (QHE) genannt. Die Entdeckung von Klaus von Klitzing eröffnete der Wissenschaft ein reiches Feld an interessanten Forschungsthemen unter anderem in der Feldkörperphysik und der Metrologie. Während heutzutage der QHE bereits zur Normung des elektrischen Widerstands verwendet wird, könnte er in Zukunft dazu verwendet werden das Kilogramm und des Ampere neu zu definieren \cite{janssen}. 