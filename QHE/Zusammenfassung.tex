Wir konnten mit dem Versuch einen guten Einblick in die Röntgenspektroskopie
gewinnen. Die charakteristischen Linien von Eisen, Molybdän und Kupfer wurden mit
recht hoher Genaugikeit nachgewiesen, wobei der größte Abstand von unseren
Bestwerten zu den Theoriewerten 
\SI{0.65}{\percent} betrag. Im Rahmen der Fehler gab es keine Abweichung. 
 Das empirische Gesetz zwischen der Intensität der
charakteristischen Strahlung und der Spannung zeigt systematische Abweichungen
für Spannungen ab \SI{30}{\kilo\volt} und sollte eher als Faustregel verstanden
werden. Das Duane-Hunt-Gesetz hingegen konnte gut bestätigt werden und hat uns
erlaubt, das Plancksche Wirkungsquantum zu bestimmen. Das Moseley-Gesetz wurde
ausführlich diskutiert und hat gute Abschätzungen für die Rydberg-Konstante
ergeben. Allerdings ist die Auswertung der \emph{Abschirmkonstante} $\sigma(Z)$
nicht wirklich sinnvoll. Mit dem Compton-Effekt konnte eine überraschend gute
Bestimmung der Compton-Wellenlänge durchgeführt werden. Eine vollständige
Aufnahme des Transmissionsspektrums von Al im gesamten Wellenlängenbereich der
Kupferanode würde helfen zu verstehen, warum die Näherung eines linearen
Spektrums solch gute Ergebnisse liefert. Die Laue-Aufnahme hat insgesamt gut
funktioniert. Allerdings könnte man die Aufhängung der Dentalfilme zum Beispiel
mit einer optischen Bank o.Ä. verbessern. Dadurch wird ein zentrales 
Auftreffen garantiert. Die Auflösung 
der Filme ist gut, eine größere Fläche wäre zwar wünschenswert, ist für die
Auswertung aber nicht unbedingt notwendig. 
